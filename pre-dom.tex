\section{Specifying the PDF DOM's Foundations}
\label{sec:specifying}

This section %
states our goals in producing a formal partial definition of PDF
(\Cref{sec:spec-goals}) and then presents the specification's core
definitions (\Cref{sec:core}), followed by its definitions of each
stage of the PDF Trust Chain (\Crefrange{sec:stage-1}{sec:stage-4}).
%
\Cref{sec:assessing} analyses the resulting specification.

\subsection{Specifying PDF: Goals and Approach}
\label{sec:spec-goals}

% REMEMBER terms: complies with standard, compatible with
Ideally, a PDF implementation should:
\begin{itemize}
\item comply with the PDF standard,
\item when not fully supporting the standard, do so gracefully,
\item support common PDF malformations
  to be compatible with extant PDFs, but \emph{without} introducing ambiguities
  or causing other unintended consequences,
\item and all the while, contain minimal bugs and stay resilient
  against all attacks.
\end{itemize}
Achieving the above will be challenging due to many factors:
\begin{itemize}
\item The intrinsic complexity of PDF:
  PDF is a \emph{less than ideal} design that reflects 27 years of
  an evolving standard.
  %
  PDF has multiple redundant features.
  %
  PDF is explicitly designed to facilitate efficient processing of
  large files (``efficiency hacks'').
  %
  PDF involves multiple sublanguages and embedded formats.
  %
  PDF contains complex parsing needs, such as backwards searching and
  embedded file offsets.
\item Lack of formality in standard. Thus, implementators
  need more effort to understand it and comply with it.
  Implementations commonly over implement, under implement,
  and wrongly implement the standard.
  Because writing a PDF implementation from scratch is a large effort,
  implementors are incentivized to patch existing code (correct or not).
\item The standard defines one thing: \emph{What is a valid PDF?},
  and leaves other decisions to the implementation:
  (1) What should absolutely \emph{not} be allowed (because in the real world
    implementations are more or less relaxed)? And similarly,
    what are deemed to be acceptable, reasonable error recovery methods?
  (2) What is required to support backwards and forwards compatibility?
  (3) What is done when redundant features are inconsistent with one
    another: which, or neither, has priority?
    Similarly, what is done when the stated PDF version and the PDF
    constructs used don't match?
  (4) What is \emph{required} from the PDF writer versus
    what do we \emph{require} the PDF reader to check?
  (5) Behavior when given in incorrect PDF.
\end{itemize}
Failure to faithfully implement the standard can result in ambiguities
between implementations as well as direct vulnerabilities (such as
Shadow Attacks or DOS attacks).

\mtnote{this last, (5), seems pretty key and
  feels like a major omission in the standard! If the answer is
  yes, then there's nothing keeping an implementation from only
  doing enough to ``work on good pdfs'' without even checking that it
  IS a good PDF.  Thus, we have the tool that only reads the 'e'
  in ``endobj''.  Is this reasonable?  Are there vulnerabilities that
  would exist in such implementations?
}

% define formal specification:
A \emph{formal specification} of a format does not immediately resolve
all of the above issues, but it is a critical artifact for clarifying
the standard, understanding the vulnerabilties, and aiding
implementors of PDF processors.
%
In this work, we have produced a specification of one component of PDF
in the Haskell programming language~\cite{jones2003haskell}.
%
A detailed presentation of Haskell's features is beyond the scope of
this paper, so we note only that it is a statically typed, pure,
functional programming language with a lazy;
%
we review its other features throughout the paper when they are
relevant.

% say more about the spec's scope:
The scope of our specification is the gap between the low level
parsers (parsing integers, parsing XRef entries, etc.) and the
processing that happens after the DOM is created (stages 5 and 6).
%
I.e., stages 1--4.
%
Primitive parsers are assumed, not included in this spec, as other
formalisms, such as the \emph{DaeDaLus Data Description
  Language}~\cite{daedalusrepo}, are better suited to specifying the
primitive parsers.
%
See \cref{sec:appendix1} for a list of the primitive parsers
with their type signatures.
%
The full specification of DaeDaLus is publicly available
online~\cite{daedalusrepo}.

\begin{itemize}
\item Our specification is formal and executable (though not
  necessarily efficient).
  %
  We already have the PDF standard: which is not always clear, not always
  consistent. 
  This is our motivation for choosing Haskell over English prose or
  pseudo-code or other informal approaches.
  For the reader
  without a reading knowledge of Haskell, we understand that parts of
  the specification could be a little obscure, but our hope is that a
  precise, formal specification may prove to be more useful than
  pseudo-code or the like!
  
\item Our specification is purely functional: no ad hoc global variables are
  hidden, the data-dependencies in the spec fully represent all the
  data-dependencies.  Motivated by the Trust Chain issues
  (\cref{sec:trust-chain}), our objective was to capture all dependencies.
  
\item Our specification hides no difficulties: one could implement a PDF parser
  by writing the omitted functions, but the spec stands complete, as
  is!
  % \footnote{One caveat: the addition of support for \emph{some} features
  % could require some re-design.  E.g., signature validation, as discussed in
  % \cref{sec:updates-and-signatures} would require non-trivial changes.
  % }
\end{itemize}

This spec supports PDF 2.0, including compressed objects and XRef streams.
%
It (safely) ignores linearization data, and in hybrid XRef PDFs
it ignores the traditional \xref{} tables designed for pre PDF 1.5 readers.
It processes signatures (as incremental updates) but it does not support
validation of signatures.

%%%%%%%%%%%%%%%%%%%%%%%%%%%%%%%%%%%%%%%%%%%%%%%%%%%%%%%%%%%%%%%%%%%%%%
\subsection{Core definitions}
\label{sec:core}
%%%% begin: Hs code not in paper %%%%
\iffalse
\begin{code}
{-# LANGUAGE EmptyDataDecls, TypeOperators, LambdaCase #-}
module Spec where
import           Control.Monad
import           Data.Char
import           Data.Foldable(foldlM)
import qualified Data.IntSet as IntSet
import           Data.List
import qualified Data.Map as M
import           Data.Map(Map)
import           Types
import           Utils
import           Primitives
import           Streams
\end{code}
\fi
%%%% end: Hs code not in paper %%%%

\lstcd{pPDFDom} (\cref{fig:spec}) implements stages 1 through 4 (illustrated in \cref{fig:pdf-trust-chain}).
% 
Line 1 contains \lstcd{pPDFDom}'s type signature, indicating that it
is a ``monadic parser'' \lstcd{P} that returns a value of type
\lstcd{DOM}.
%
In general, monads are a rich class of parameterized types that
describe a surprising variety of data containers.
%
For the purposes of this paper, it suffices to note that:
\begin{itemize}
  \item A monad is a way to sequence effectful constructs in a purely
    functional manner; effects can be global variables (or mutable
    state), exceptions, I/O, and etc.
  \item We use the \lstcd{P} monad
    that captures, under the hood, our desired effects.
  \item The \lstcd{P} monad supports
    \begin{itemize}
    \item one read-only variable (the PDF file being read)
    \item one mutable variable \rdloc{}, the offset in the file where
      reading (i.e., the primitive parsing) occurs; we update it
      with
  \begin{codeNoExecute}
  seekPrimitive :: Offset -> P ()
  seekPrimitive off = <update readLocation with 'off'>
  \end{codeNoExecute}
      \mttodo{readLocation equals \rdloc{}?}  and it is accessed
      implicitly by the primitive parsers in \lstcd{P}.
    \item Exceptions (i.e., the \lstcd{Maybe} monad): any action
      may fail, failure propagates naturally without the need for
      throwing/catching exceptions.
    \end{itemize}
\item We use the \lstcd{do} syntactic sugar to easily sequence
  monadic actions.
  E.g.,
  \begin{codeNoExecute}
  topAction :: P Int
  topAction = do
              result1 <- action1 args1...
              let x = <PURELY-FUNCTIONAL-EXPRESSION>
                  f x = <FUNCTION> -- not an action
              result2 <- action2 result1 args2...
              return (anyPureFunction result2)
  \end{codeNoExecute}
  The term ``action'' refers to a monadic function: i.e., a function
  having a type scheme of ``\lstcd{a -> P b}''.
\item Important note: any action can fail and stop the parsing,
  and the use of the \lstcd{x <- expr} ``bind'' syntax indicates
  that \lstcd{expr} is an action.
\end{itemize}

To further elucidate the code, we sometimes add \emph{type
  annotations} on expressions in the code (see \cref{fig:spec} lines
8, 11, and 14).

\begin{figure}[t]
\centering
\lstset{numbers=right}
\begin{code}
pPDFDom :: P DOM
pPDFDom =
    do
    -- Stage 1:
    (seek,xrefOffset,version) <- findHeaderAndTrailer
    -- Stage 2:
    updates <- parseAllIncUpdates seek xrefOffset
               :: P [(XRefRaw, TrailerDict)]
    -- Stage 3: combine all the updates to get a single xref table
    xrefs <- combineUpdates seek updates
             :: P (Map ObjId (Offset :+: Type2Ref))
    -- Stage 4:
    dom <- transformXRefMapToObjectMap seek xrefs
           :: P DOM
    -- final checks and validations:
    finalValidations dom version updates
    return dom
\end{code}
\caption{A formalization of stages 1--4.}
\label{fig:spec}
\end{figure}


%%%%%%%%%%%%%%%%%%%%%%%%%%%%%%%%%%%%%%%%%%%%%%%%%%%%%%%%%%%%%%%%%%%%%%
\subsection{Stage 1: Find and Parse the Header and Trailer}
\label{sec:stage-1}

% give the code for stage 1:
The definition of stage 1 is as follows:
\begin{code}
findHeaderAndTrailer :: P (SEEK, Offset, Version)
findHeaderAndTrailer =
    do
    -- find "%PDF-x.y" near start of file, searching the first 1000 bytes:
    (version, headerOffset) <- findPDFHeader 1000

    -- search backwards from EOF for 'startxref', gives up after 1001 bytes:
    xrefOff <- findStartxrefThenParseToEOF 1001
    
    let seek n = seekPrimitive (headerOffset+n)
    return (seek, xrefOff, version)
    
type SEEK = Offset -> P () -- type of seek
\end{code}

Neither of the magic numbers $1000$ or $1001$ are specified precisely
by the PDF standard, which requires only that they be numeric values
that are ``reasonably small'' \todo{is this a quote from the standard?}.
%
This is an immediate cause of ambiguity in the standard (and thus an
invitation for \pd{} in conformant document processors).

File offsets in a PDF are not defined in relation to the beginning of
the file, as might be expected, but instead to the beginning of the
\emph{PDF header};
%
the \lstcd{seek} provides a clean abstraction that defines this
aspect.
%
But we will need to pass \lstcd{seek} to all the actions that need to
``seek'' in the PDF file (note this being done in \cref{fig:spec}).

Note that the two function calls (lines 5 and 8) have no data
dependencies, thus they could be written in either order.
\haskellnote{The spec has no global variables beyond \rdloc{} and neither function
  depends on it.}

The code for \lstcd{findStartxrefThenParseToEOF} is
omitted, as it is more tedious than instructive.
%
Informally, it:
\begin{enumerate}
\item Finds the ``EOF marker'' \lstcd{\%\%EOF} (near the end of the physical
  file);
\item Parses ``backwards" to find the last \lstcd{startxref} keyword
  followed by an end-of-line sequence;
\item Parses the integer value encoded as a sequence of ASCII bytes
  (this represents the byte offset in the PDF file, which as noted
  above is adjusted for any preamble to the Header).
\end{enumerate}

%%%%%%%%%%%%%%%%%%%%%%%%%%%%%%%%%%%%%%%%%%%%%%%%%%%%%%%%%%%%%%%%%%%%%%
\subsection{Stage 2: Find and Parse Incremental Updates}
\label{sec:stage-2}

We have not yet parsed the trailer information, but we know where
it starts.  Parsing the final XRef table and Trailer dictionary are
done using this function:

\lstset{numbers=right}
\begin{code}
type Update = (XRefRaw,TrailerDict)

pIncUpdate_Traditional :: SEEK -> P Update
pIncUpdate_Traditional seek =
    do
    (xrefRaw, xrefEndOff) <- pXrefRaw :: P (XRefRaw,Offset)
    validate $
      verifyXrefRaw xrefRaw
        -- this ensures no duplicate objectIds
    seek xrefEndOff
       -- This seek is needed because pXrefRaw doesn't need to read
       -- the entries of each XRef subsection, so let's leave the
       -- current file read location after the xref table.

    pSimpleWhiteSpace -- no comments allowed between XRef table and ..
    keyword "trailer"
    trailerDict  <- pDictionary
    trailerDict' <- dictToTrailerDict trailerDict
                    -- ensures dictionary has proper keys
    return (xrefRaw, trailerDict')
\end{code}
\lstset{numbers=none}

Note \lstcd{validate} on line 7: this is a special function that we
use to demarcate semantic checks (line 8) that are not necessary to
create the DOM but which could detect invalid or inconsistent PDFs.
%
The check could be used in a ``validate'' mode, but we wouldn't expect
all implementations to necessarily implement the check.

We have intentionally designed this to return \lstcd{XRefRaw} objects,
% TODO: show defn?
i.e., we find the subsections of the \xref{} table, but we do not
parse the sequence of \xref{} entries in each subsection.
(We will describe the motivations for this after
we have explored stage 3.)
%
Note that, even for \lstcd{XRefRaw},
we know the set of \objid{}'s and for each we can find the offset
of its \xref{} entry.  This design relies on implementing
the specification strictly: an \xref{} entry must have exactly 20 bytes.

However, parsing the \xref{} table and trailer is just a special case
of parsing \emph{all} the incremental updates, which we show here:
\lstset{numbers=right}
\begin{code}
parseAllIncUpdates :: SEEK -> Offset -> P [Update]
parseAllIncUpdates seek offset =
  parseAllIncUpdates' IntSet.empty seek offset

parseAllIncUpdates' :: IntSet.IntSet -> SEEK -> Offset -> P [Update]
parseAllIncUpdates' prevSet seek offset =
    if offset `IntSet.member` prevSet then
      error "recursive incremental updates"
    else
      do
      seek offset
      (xref,trailerDict) <- pIncUpdate seek
      case trailerDict_getPrev trailerDict of   -- lookup 'Prev' key
        Nothing      -> -- no Prev key, we're done:
                        return [(xref,trailerDict)]
        Just offset' -> -- Prev key found, find updates starting at
                        -- offset':
                        do
                        us <- parseAllIncUpdates'
                                (IntSet.insert offset prevSet)
                                seek
                                offset'
                        return ((xref,trailerDict):us)
\end{code}
\lstset{numbers=none}

Half the complexity of the above code is to ensure that loops in
the \lstcd{Prev} offsets don't result in the code going
into an infinite loop: we keep track of the offsets of every update
we have processed in the set \lstcd{prevSet}.
What we are doing here is
(1) \lstcd{seek}-ing to the \lstcd{offset} and parsing the update;
(2) checking for a \lstcd{Prev} key, if no key we're done, otherwise
    continue.

Also, since we support both traditional (pre 1.5 PDF) and stream
\xref{} tables:

\begin{code}
pIncUpdate :: SEEK -> P (XRefRaw,TrailerDict)
pIncUpdate seek =
      pIncUpdate_Traditional seek
  .|. pIncUpdate_XrefStream seek
      -- I.e., parse one or the other,
      -- syntactically, they are mutually exclusive.

pIncUpdate_XrefStream :: SEEK -> P (XRefRaw,TrailerDict)
pIncUpdate_XrefStream seek = notImplementedYet
\end{code}

The function \lstcd{pIncUpdate_XrefStream} is more
complex than \lstcd{pIncUpdate_Traditional} but the resulting
data is equivalent.  It also allows for determining the subsections
without parsing the \xref{} entries.

So, here's where we are at this point in the computation:
\begin{itemize}
\item We can determine all of the \objids{} in the PDF.
\item We have not parsed any PDF values except trailer dictionaries.
\item We have definitely rejected some PDFs by this point.
\item We could output the trailer dictionaries or the list of
  incremental updates.
\item We have detected overlapping \objids{} in the \xref{} tables.
\end{itemize}


%%%%%%%%%%%%%%%%%%%%%%%%%%%%%%%%%%%%%%%%%%%%%%%%%%%%%%%%%%%%%%%%%%%%%%
\subsection{Stage 3: Combine Incremental Updates}
%
Updates are processed from last (near EOF) to first (near the file's beginning):
\begin{code}
combineUpdates :: SEEK
               -> [(XRefRaw, TrailerDict)] 
               -> P (Map ObjId (Offset :+: Type2Ref))
combineUpdates seek updates =
    do
    let (xref,dict):us = updates -- safe because: null(updates)==False

    -- parse all xref entries in last (near EOF) update:
    xrefEntries <- mapM (thawXRefEntry seek xref) (getObjIds xref)

    -- initial Map:
    let tbl0 :: Map ObjId XRefEntry
        tbl0 = M.fromList xrefEntries

    -- merge each update into tbl0:
    tbl1  <- foldlM (mergeUpdate seek) tbl0 us
    return (removeFrees tbl1)

type XRefEntry = Free :+: (Offset :+: Type2Ref)
\end{code}

Each incremental update can add new objects, mark existing in-use objects as free, or update objects.
%
\lstcd{XRefEntry} is defined to process updates from last to first;
%
it could have been defined alternatively to parse \xref{} entries only as needed.
%
\todo{why did we choose the option that we did?}

\begin{code}
mergeUpdate :: SEEK
            ->  (Map ObjId XRefEntry)
            -> (XRefRaw, TrailerDict)
            -> P (Map ObjId XRefEntry)
mergeUpdate seek map0 (xref,dict) =
    do
    let objIds       = getObjIds xref
        neededObjIds = objIds \\ M.keys map0
                       -- set subtraction

    -- only parse (thaw) needed XRef entries:
    newEntries <- forM neededObjIds
                       (thawXRefEntry seek xref)
    return (M.union map0 (M.fromList (newEntries :: [(ObjId,XRefEntry)] )))

data Free = Free  -- currently we are ignoring free objects
\end{code}
Note the type signatures of functions we are not showing:
\begin{code}
removeFrees :: Map k XRefEntry -> Map k (Offset :+: Type2Ref)

thawXRefEntry :: SEEK -> XRefRaw -> ObjId -> P (ObjId, XRefEntry)

getObjIds :: XRefRaw -> [ObjId]
\end{code}

For simplicity, we have neglected to use the \lstcd{Size} key (from
the trailer dictionaries) to constrain \objids{}.  Note these comments
about \lstcd{Size} from the standard:
\begin{itemize}
  \item Any object in a cross-reference section whose number is
    greater than this value shall be ignored and defined to be missing
    by a PDF reader.
  \item Equivalently, this value shall be 1 greater than the highest
    object number defined in the PDF file.
\end{itemize}

Not only is this stage already complex, the
possible---semantics changing---variations on
\lstcd{mergeUpdate} are surprisingly many:
\begin{itemize}
\item There appear to be multiple ways to enforce \lstcd{< Size} as we
  process individual updates.
\item When all objects are defined (we can determine this using
  \lstcd{Size}), we could stop processing previous updates.
\item There are many \lstcd{validate} instances we might add,
  some being unclear even what the standard would enforce:
  \begin{itemize}
  \item prohibit double frees
  \item prohibit the update of unfree objects
  \item prohibit updates that don't increment the generation number
  \item prohibit freeing of non-existent \objids{}.
  \item prohibit ``unconvential use'' of generation numbers.
  \item validate that the offset of new objects in the update are
    defined in the ``body region'' of the respective update (not point
    to previous nor subsequent regions)
  \end{itemize}
\end{itemize}
%
None of the above properties can be validated upon the completion of Stage 3.
%
Given that some of the properties may be indicators of shadow attacks or PDFs that are otherwise suspicious, Stage 3 is thus critically important as a stage for security validation.

As promised, we'll now discuss the separation of Stage 2 from Stage 3
and this notion of \lstcd{XRefRaw}: we \emph{could} have merged the
stages, but at the expense of forcing every implementation to do more
parsing than it strictly needs to.
%
While Stages 2 and 3 could feasibly be defined as a single stage, it would effectively force every compliant to parse and validate to a degree that may not be needed by many applications.
%
Our specification was designed to include as much validation 

Our philosophy here is to capture
a reasonably realistic implementation with our specification and put
functionality that is in the realm of validation into \lstcd{validate}
instances.


%%%%%%%%%%%%%%%%%%%%%%%%%%%%%%%%%%%%%%%%%%%%%%%%%%%%%%%%%%%%%%%%%%%%%%
\subsection{Stage 4: Transform XRef Map to Object Map (DOM)}
\label{sec:stage-4}

Getting this stage right was a key motivation for writing our
specification, and it turned out to be surprisingly complicated.
\begin{code}
transformXRefMapToObjectMap
  :: SEEK -> Map ObjId (Offset :+: Type2Ref) -> P DOM
transformXRefMapToObjectMap seek xrefs0 = do
\end{code}
It has three separate stages all of which do some reading/parsing of
the file. Stage 4.1
transforms \lstcd{xrefs0} into \lstcd{xrefs1}, note the types:
\begin{codeNoExecute}
  xrefs0 :: Map ObjId (Offset               :+: Type2Ref) 
  xrefs1 :: Map ObjId (TopLevelDef_UnDecStm :+: Type2Ref)
\end{codeNoExecute}
\haskellnote{
We are using the infix type operator \lstcd{a :+: b} as a synonym for
\lstcd{Either a b}, \lstcd{Either} is the Haskell sum type having two constructors
\lstcd{Left} and \lstcd{Right}.
}
%
We look up traditional \lstcd{XRef}'s and parses the objects,
see line 3
\begin{code}
    -- Stage 4.1: parse all uncompressed objects
    xrefs1 <- mapM
                (mMapLeft (\o-> do {seek o; pTopLevelDef_UnDecStm}))
                xrefs0
\end{code}
We can't do more in stage 4.1 because we can't yet decode
object streams nor can we decode streams: in both cases these top
level objects \emph{cannot} always be parsed without looking up
integers in the DOM: as \lstcd{Length} keys and similar might contain
indirect references to top level integers.
\keypoint{
  In PDF we need to lookup and parse some objects in the DOM
  before we can parse other objects in the DOM!}
% TODO: ^ rewrite to make clearer.

Stage 4.2 further refines the xref map, computing the second from the first
\begin{codeNoExecute}
  xrefs1 :: Map ObjId (TopLevelDef_UnDecStm :+: Type2Ref)
  xrefs2 :: Map ObjId (TopLevelDef          :+: Type2Ref) 
\end{codeNoExecute}
in this code
\begin{code}
    -- Stage 4.2: decode stream bytes, pre-process ObjStm streams
    xrefs2 <- mapM
                (mMapLeft (extractStreamData xrefs1))
                xrefs1
\end{code}
We have a sufficiently complete \lstcd{DOM} in \lstcd{xrefs1} in which
we can lookup the integers that are needed to finish decoding the
streams we left undecoded.
If we cannot find the referenced integer in the DOM, we have a true PDF
error because any indirect \lstcd{Length} values are required to be at the
top level.

At this point, in \lstcd{xrefs2}, we still have \lstcd{ObjId}'s that point
(via \lstcd{Type2Ref}) into \lstcd{ObjStm} streams.
Only in stage 4.2 were we able to decode the stream inside of
which is the object to be parsed.
% TODO this last text is rather complicated/tedious!

\mttodo{bring in some of the Alg. Data Type defs?}

So at this point, (1) we can compute body cavities \todo{define this} as
we can parse all the top level objects, (2) object streams have been
pre-processed, but we have not yet parsed the objects inside them.

Finally, stage 4.3 computes \lstcd{domCandidate} from \lstcd{xrefs2}
\begin{codeNoExecute}
  xrefs2       :: Map ObjId (TopLevelDef :+: Type2Ref) 
  domCandidate :: Map ObjId TopLevelDef
\end{codeNoExecute}
using the same pattern we've been seeing but now we no longer have 
the sum, we have a mapping from \lstcd{ObjId}'s to PDF Values:
\begin{code}
    -- Stage 4.3: resolve Type 2 references
    domCandidate <- mapM
                     (return `either` derefType2Ref xrefs2)
                     xrefs2
    return domCandidate
\end{code}

At this point every object referenced via the XRef table has been
correctly parsed. However, (1) object definitions in the PDF body
area that are not ``XRef referenced'' are neither read nor parsed,
% 
(2) objects inside an \lstcd{ObjStm} stream that are not ``XRef
referenced'' are also neither read nor parsed.

Note that the \emph{Catalog Dictionary} (which has the optional
\lstcd{Version} key) may have been in an Object stream, so in general
we may not know till this point which version of PDF we are
parsing\footnote{We wonder if this was the intention of the Standard!}.


%%%%%%%%%%%%%%%%%%%%%%%%%%%%%%%%%%%%%%%%%%%%%%%%%%%%%%%%%%%%%%%%%%%%%%
\subsection{Further Validation}
\label{sec:validating}

\mttodo{remove or fix the etc stuff}

We have a few checks and validations that we are unable to do until
the DOM is created:
\begin{code}  
finalValidations dom version updates =
    do    
    version' <- updateVersionFromCatalogDict dom version
    let etc = stub "extra keys in the trailer dict" :: Dict -- TODO!

    if version' > (2,0) then
      warn "PDF file has version greater than 2.0"
    else
      -- version' <= (2,0)
      when (not (null etc)) $
        warn "trailer dictionary has unknown keys (per PDF 2.0)"
    validate $
      versionAndDomConsistent version' dom
    
    validate $
      trailersConsistentAcrossUpdates updates
\end{code}

Due to PDF-1.5 additions, we are now in the odd position that we
cannot determine the PDF version until after we have created the DOM.
So although we
can check that the created \lstcd{dom} is consistent with
\lstcd{version'},
% 
we cannot use \lstcd{version'} to validate any of the other processing
in stages 1--4.
%
For instance, one would like to verify that Object Streams are not
used when the version is PDF-1.4 or earlier.
%
Such a check would be possible if we were to update the spec to pass
more information through the stages so we could verify more after
computation of the DOM.

\section{Assessing The Specification}
\label{sec:assessing}
%

%%%%%%%%%%%%%%%%%%%%%%%%%%%%%%%%%%%%%%%%%%%%%%%%%%%%%%%%%%%%%%%%%%%%%%
\subsection{Comparison with a definition containing only one stage}
\label{sec:single-pass-problems}

%%%%%%%%%%%%%%%
\begin{figure}[t]
  \centering
  \begin{myc}
    // xref and dom start with nothing in them:
    //  - in reality, their sizes would be dynamically allocated.
    XREF_ENTRY xref[Size];     
    DOM_ENTRY  dom[Size];
    
    // 'dom' is updated dynamically, on demand, via 'deref':
    PdfValue deref(ObjId oi) {
      if (evald(dom[oi]))
        return dom[oi];
      else if (infiniteloopdetected())
        quit ();      
      else {
        o = lookupOffsetInXref(oi); // updates xref[oi]
        seek(o);
        v = parseObject();
        dom[oi] = v;
        return v;
      }
      
    PdfValue parseObject (){
      ...
      if (some Stream object) {
        ...; len = deref(oi'); ... // need this to finish parsing!
      }
      ...
    }
    
    OFFSET lookupOffsetInXref(oi) {
      if xref_evald(xref[oi]) then
        return xref[oi];
      else {
        // follow Prev pointers if not in top xref
        //  - make sure no infinite loop in chasing Prev's
        /* ... */
        xref[oi] = offset;
        return xref[oi];
      }
    }
  \end{myc}
  \caption{\dsp{}, A Single Stage, Imperative Parser.}
  \label{fig:dsp}
\end{figure}
%%%%%%%%%%%%%%%

One might ask if it's possible to write a single stage version of the
specification.  I.e., Can we do without so many stages?
The answer is \emph{Yes, but at a cost we don't want to pay!}
Let's refer to the our multi-stage specification as the \ssp{}
(staged) specification.
%
We will compare it to a single-stage specification\footnote{
  Which will not be so clear, but we see no way around this.
}, the \dsp{}
(dynamic) specification in which stage 1 is the same
but stages 2, 3, and 4 are
merged together.
%
In our experience, PDF tools generally work more like \dsp{} than \ssp{}.

\dsp must be understood imperatively; see \cref{fig:dsp} for a sketch
of a possible implementation in C-like notation.
The key things to note about \dsp{} are
\begin{itemize}
\item \lstcd{XREF_ENTRY} and \lstcd{DOM_ENTRY} are both types that mutate progressively
   from unparsed/unknown to fully evaluated.
\item \lstcd{deref()} and \lstcd{parseObject()} are mutually recursive functions.
\item implementing \lstcd{infiniteLoopDetected()} is non-trivial.
\end{itemize}
Also to note about \dsp{}:
\begin{itemize}
\item it is \emph{not} equivalent to \ssp{}: \dsp{} potentially reads
  less of the input file and accepts more input files.  it is
  naturally ``lax'': it would for instance allow a \lstcd{Length} to
  be stored in an \lstcd{ObjStm} stream as long as an infinite loop
  wasn't detected.  It would be possible to extend (and complicate) \dsp{}
  to approximate \ssp{} better.
  % E.g.,
  %  - have a =derefLength= / =derefFromUncompressed=
  %  - More complicated than just this, because this won't catch error if we
  %    luck out and when we request the length it is already decoded.
\item it is nicely lazy if the PDF tool doesn't need to \lstcd{deref}
  every object identifiers, even more lazy than \ssp{}.
\end{itemize}

We believe that \ssp{} is clearly preferable to \dsp{} for these two
simple reasons:
\begin{itemize}
\item It corresponds to the standard, and does so obviously.
\item It does not use general recursion, so it is obvious that \ssp{}
  algorithm terminates on all inputs.
\end{itemize}
\ssp{} gives us a few other advantages over \dsp{}:
\begin{enumerate}
\item The functional, declarative, and typed structure enables
  us to understand the stages conceptually.  (Even if one chooses
  not to implement in a like manner.)
\item It demonstrates the non obvious fact that one can
  implement stages 2-4 and keep the ``state of evaluation'' of each
  object identifier the same in each pass.
\item We know exactly what is and isn't parsed, regardless of the
  order in which one traverses the \lstcd{xref}.
\item It is intrisically parallelizable due to the extensive use of
  map-like combinators.
\item The declarative nature of \ssp{} makes it very amenable to
  modification: e.g., the simple addition of the \lstcd{validate}
  operator.
\end{enumerate}


%%%%%%%%%%%%%%%%%%%%%%%%%%%%%%%%%%%%%%%%%%%%%%%%%%%%%%%%%%%%%%%%%%%%%%
\subsection{Incremental Updates and Signatures: What A Tangled Web}
\label{sec:updates-and-signatures}

PDF supports the ability to use digital signatures to sign documents
as part of the PDF standard itself.  Signed documents can
subsequently be signed or ``incrementally updated,'' repeatedly.

Digital signatures are just a special form of incremental update, and
thus any PDF reader can just ignore the signature and process it as a
trivial update.  This allows readers that don't support signatures to
elegantly ignore them.

However, when we \emph{do} have a digital signatures and a supporting
reader, things get complicated, as evidenced by Shadow Attacks
\cite{mladenovTrillionDollarRefund2019,ndsssymposiumNDSS2021Shadow2021}.
%
To validate a PDF with a digital signature entails identifying at
which iteration of the PDF document the digital signature was applied
and then validating the digital signature in the context of that
specific DOM and the objects that were in effect at that instance in
time.

But this method completely \emph{breaks} the incremental update
abstraction.  As we discussed, stages 2 and 3 handle all
the incremental update processing and then the information is gone as
it is unneeded by subsequent stages.
% 
This was a reasonable design until ``incremental updates'' were
inadvisedly repurposed for signatures.  Now an incremental update has
semantic content, and a reader supporting validation and display of
signed PDFs will need to keep track of the contents of the DOM at each
signature.

Our specification does \emph{not} support signature validation or
display, and it would be a non-trivial effort to do this.  However,
this complication is going to be reasonable with our \ssp{} approach;
whereas it is unclear whether the \dsp{} approach is able to support
this at all.


%%%%%%%%%%%%%%%%%%%%%%%%%%%%%%%%%%%%%%%%%%%%%%%%%%%%%%%%%%%%%%%%%%%%%%
\subsection{Analysis of results}
\label{sec:results-analysis}

% claim conformance to the spec:
The specification as presented adheres to the PDF 2.0 standard.
% bugs found in the standard:
In the process of producing the format definition described above, we
raised multiple issues concerning the PDF standard with the
ISO~\cite{isotc171sc2wg8ISO32000220202020}, specifically an issues
concerning the definition of cross-reference streams.
%
In the process, we raised several other issues related to the
definition of cross-reference streams that were acknowledged by PDFA
and have since been resolved, including %
\textbf{(1)} requirements for object numbers in cross-reference
subsections~\cite{pdfIssue146} and %
\textbf{(2)} corrections to the denotation of the \texttt{XRefStm}
field of a trailer dictionary~\cite{pdfIssue147}.

\todo{Peter: confirm that these issues are related to the paper}

% handling de-facto specs:
The specification's conformance to the standard arguably
limits its applicability given that many PDF tools allow multiple
variances from the standard.
%
However, the opportunity for us is to use our specification to encode
some of these common extensions and to explore if the combination of
these extensions are truly unambiguous.

With our specification based approach, it would not be onerous
to refactor our specification to support different PDF variations:
\begin{itemize}
\item strict PDF-2.0.
\item strict PDF-2.0, validating everything.
\item PDF-2.0 with some common extensions that are determined to be unambiguous.
\end{itemize}
%
\todo{what is it about the spec that would make this easy?}

% discuss concision:
Our specification owes its concision to multiple factors:
\begin{enumerate}
\item our intention to make it as clear as possible,
\item ignoring some ``engineering'' aspects (e.g., informative error
   messages, recovery, etc.), and
\item omitting code for some of the simple, tedious functions.
\end{enumerate}
The current specification could be extended to form a complete
\emph{reference implementation} by extending it to include
%
\textbf{(1)} implementations of relatively minor features that were
not the focus of the current work;
%
\textbf{(2)} implementations of all parsers referenced as primitives; and
%
\textbf{(3)} computation that generates the document's DOM.

% future work:
Future work might also involve adding further validation checks,
including %
\textbf{(1)} processing linearization data and ensuring consistency
with the non-linearized DOM; %
\textbf{(2)} processing both ``branches'' of a hybrid PDF and ensuring
some form of constency between pre 1.5 readers and 1.5+ readers; %
\textbf{(3)} adding signature validation, see
\cref{sec:updates-and-signatures}.


%%%%%%%%%%%%%%%%%%%%%%%%%%%%%%%%%%%%%%%%%%%%%%%%%%%%%%%%%%%%%%%%%%%%%%
%%%% not implemented functions %%%%%%%%%%%%%%%%%%%%%%%%%%%%%%%%%%%%%%%
%%%% begin: Hs code not in paper %%%%
\iffalse
\begin{code}

trailersConsistentAcrossUpdates :: [Update] -> Bool
trailersConsistentAcrossUpdates = stub

-- signatures above:
removeFrees m = M.mapMaybe
                  (\x -> case x of Right y -> Just y
                                   _       -> Nothing)
                  m
thawXRefEntry = notImplementedYet
getObjIds = notImplementedYet

\end{code}
\fi
%%%% end: Hs code not in paper %%%%

