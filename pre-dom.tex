% ------------------------------------------------------------------------------
\section{Specifying the DOM's Foundations \note{4pp}}
\label{sec:specifying}

\todo{somewhere we need to compare with the monolithic approach that APPEARS
  to not have multiple phases.}

\subsection{[Motivating Specification]}
% REMEMBER: [terms: complies with standard, compatible with]

\begin{lstlisting}[style=meta]
- an implementation
  - should follow the standard
  - should safely support less than standard
  - pragmatically support some common extant data malformations
  - should carefully support more than the standard
  - should not "inf. loop"
    - lots of opportunities - failure to notice digitally signed PDFs that have been tampered! 
      - elaborate?
\end{lstlisting}

\begin{lstlisting}[style=meta]
- Lack of formality in standard. Thus, implementations:
  - are more effort
  - over implement, under implement, wrongly implement
  - backwards and forwards compatibility
  - "backwards parsing"
  - some requirements will not be checked by PDF readers ("writer only" file requirements) 
  - patch existing vs implement from scratch
- No definition of acceptable, reasonable error recovery
- Less than ideal design that reflects 27 years of an evolving standard
- Pre-DOM processing
  - is where many parsing errors & recovery occur
  - is non-trivial
  - involves multiple interdependent features and subtle dialects
  - involves multiple redundant features
    - schizophrenic if these features aren't mutually consistent
\end{lstlisting}

\subsection{The pre-DOM constructs}
\todo{Hmmm: how much detail to go into?
      How will reader understand next section if we say nothing?
}

\subsection{Specifying pre-DOM components}
\begin{lstlisting}[style=meta]
- presented
- going into PDF details, as needed (this section or separate section?)
\end{lstlisting}
