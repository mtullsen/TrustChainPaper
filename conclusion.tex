% ------------------------------------------------------------------------------
\section{Conclusion \note{1.5pp}}
\label{sec:conclusion}

\subsection{Contributions}

\todo{prosify the following bullet points}
\begin{lstlisting}[style=meta]
  - Clarified aspects of PDF with respect to incremental updates,
    minor parsing details.
  - Submitted a problem in the definition of cross reference streams
    (\cite{isotc171sc2wg8ISO32000220202020} Sec 7.5.8) to
    ISO via Peter Wyatt.
  - Other ``submissions'' to ISO via Peter ...
    - see various pdf-issues in GitHub and changes published at https://pdf-issues.pdfa.org/
  - Pre-DOM specification
    - dealing with parsing (DOM-dependent) object streams
      (the lack of clarity as submitted to ISO)
  - Tool for inspecting and checking PDF at the pre-DOM level:
    Created tool for exploring the DOM Antecedent structures
    as well as validating them (more than a
    PDF reader necessarily does).
    - Based on Galois's \todo{TA2} PDF parser, this tool can
      parse and validate each incremental update separately
      display "incremental updates," "incremental xref tables,"
      parsed objects, and cavities (bytes that are not used)
      validate that object definitions do not overlap (in their source bytes)
  - Generic and principled approach for dealing with
    \note{complex,onery,etc} formats.
    - E.g., cavities, embedded offsets/lengths, etc
\end{lstlisting}

\subsection{Applications to Other Formats}

\todo{...}

\subsection{Future Work}

Support more features in our specification:
\begin{lstlisting}[style=meta]
  - linearized PDFs
  - specify hybrid xref PDFs
  - add further ``validators''
    - no cavities
    - no dead objects
    - etc.
  - add ``Phase X'' DOM validation
\end{lstlisting}

Add more features to our tool and specification:
\begin{lstlisting}[style=meta]
  - support linearized files (to improve cavity detection)
  - more consistency checks: e.g, for hybrid xref files
  - further analysis and categorization of cavities
\end{lstlisting}

\todo{And even more?}
