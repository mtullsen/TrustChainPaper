\section{Related Work}
\label{sec:rel-work}

% defining the PDF standard:
The PDF Association maintains the primary effort to specify the PDF
format, which is done informally in the PDF standard ISO 32000, and has now
been backed partially be a machine-readable definition of the Document
Object Model~\cite{peterwyattArlingtonPDFModel2021}.
%
The Caradoc project aimed to define a machine-readable specification
of PDF with a verified parser, using the Coq interactive theorem
prover~\cite{g.endignouxCaradocPragmaticApproach2016}.
%
Caradoc demonstrated that a small---but meaningful---core subset could
be formalized in Coq;
%
however its definition does not include a complete definition of many
of the key features used to define referential context, including
cross-reference streams and incremental updates.

% DOM/security attacks:
A wealth of recent work has discovered methods (named \emph{Shadow
  Attacks}) for subverting the end-to-end guarantees of document
integrity that are provided by digital signatures in
PDF~\cite{mullerPracticalDecryptionExFiltration2019,mladenovTrillionDollarRefund2019,mullerProcessingDangerousPaths2021,ndsssymposiumNDSS2021Shadow2021,rohlmannBreakingSpecificationPDF2021}.
%
Many of these attacks rely on mutating the referential structure of a
document by updating its cross reference table via incremental
updates.
%
These attacks published to date only perform basic
updates to the DOM, and do not rely on nuanced features of semantics
of cross-reference streams and object streams.
%
Recent work~\cite{itextShadowAttack} has also introduced automated
tools for determining if a given PDF document may likely be extended
to form a Shadow Attack, which perform an analysis of a given
document's DOM to identify patterns that indicate that the document
has likely been crafted in order to perform a Shadow Attack.
%
The search for such patterns does not involve a rigorous definition of
data structures used to refer to objects in the DOM, which is provided
by our approach.

% parser combinators:
Parser combinators are a well-studied abstraction for constructing
parsers programmatically and are available as libraries for several
industrial-strength languages, including the \texttt{parsec}
library~\cite{leijen2001parsec} originally implemented for Haskell and
ported to go, F\#, C\#, and Java;
%
the \texttt{nom} parser combinator library in Rust can be used to
parse messages with \emph{zero copies}~\cite{couprie2015nom}.
%
Several experimental data-description languages, such as
Parsley~\cite{mundkurResearchReportParsley2020} and
hammer~\cite{bratus2017curing}.
%
Many of these libraries provide constructs for explicitly capturing
and setting input streams;
%
thus, we expect that many of them could be used to define the aspects
of the PDF related to referential context that we have explored in
this work.
%
We believe that this and related aspects of the PDF definition will
provide compelling motivating case studies in the design of these
languages that will drive studies of how they can used to define the
features precisely and completely.
