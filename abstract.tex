% abstract:
\begin{abstract}
  % context:
  Conventional data-description languages, such as context-free
  grammars, naturally define formats in which the semantic
  interpretation of an input segment can depend hierarchically on
  the semantic interpretations of its sub-segments.
  %
  They cannot be applied to many practical and security-critical
  formats---including the Portable Document Format (PDF)---in which
  the interpretation of a segment as a \emph{Document Object Model
    (DOM)} graph depends on a concept of reference and complex
  contextual data that binds references to data objects.

  %
  Such referential context itself is defined
  discontinuously, and is often compressed, to satisfy practical constraints on
  usability and performance. The integrity of these references and their context must be ensured 
  so that an unambiguous DOM graph is established from a basis of trust.

  % result:
  This paper describes a case study of a critical instance of such a
  design, namely the construction of PDF \emph{cross-reference tables},
  in the presence of potentially multiple incremental updates and multiple 
  complex dialects expressing these references.
  %
  Over the course of our case study, we found that the full definition of
  cross-reference data in PDF contains several subtleties that are
  interpreted differently by natural implementations, but which can
  nevertheless be formalized using parser combinators in a
  data-definition language with constructs for explicitly capturing
  and updating input streams.
  %
  Our definition has served as a foundation for implementing format
  security analyses that arise naturally while instantiating and constructing DOM's,
  including the analysis of document \emph{cavities} that may store
  unexamined data used to construct file \emph{polyglots}.

\end{abstract}
