% abstract:
\begin{abstract}
  % context:
  Conventional data-description languages, such as context-free
  grammars, naturally define formats in which the semantic
  interpretation of a large segment of input depends hierarchically on
  the semantic interpretations of its sub-segments.
  %
  They cannot be applied to many practical and security-critical
  formats---including the Portable Document Format (PDF)---in which
  the interpretation of a segment as a \emph{Document Object Model
    (DOM)} graph depends on a concept of reference and complex
  contextual data that binds data objects to references.
  %
  Such referential context itself must often be defined
  discontinuously and compressed, to satisfy practical constraints on
  usability and performance.

  % result:
  This paper describes a case study of a critical instance of such a
  design, namely the construction of PDF \emph{cross-reference tables},
  in the presence of incremental updates and cross-reference streams.
  %
  Over the course of case study, we found that the full definition of
  cross-reference tables contains several subtleties that could be
  mis-implemented by natural implementations, but which can
  nevertheless be formalized using parser combinators in a
  data-definition language with constructs for explicitly capturing
  and updating input streams.
  %
  Our definition has served as a foundation for implementing format
  security analyses that arise naturally in the presence of DOM's,
  including the analysis of document \emph{cavities} that may store
  unexamined data used to construct file \emph{polyglots}.
\end{abstract}
