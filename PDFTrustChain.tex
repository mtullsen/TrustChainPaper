\documentclass[conference,12pt]{IEEEtran}
% Re-allow e.g. the \thanks command.
\IEEEoverridecommandlockouts

% to be able to draw some self-contained figs
\usepackage{tikz}
\usepackage{amsmath}
\usepackage{minted}
\usepackage[utf8]{inputenc}
% \usepackage{todo}
\usepackage{csquotes}
\usepackage{amsthm}
\usepackage{siunitx}
\usepackage[caption=false]{subfig}
\usepackage{listings}
\lstset{
  basicstyle=\scriptsize\ttfamily,
  mathescape=true,
  %numbers=left,
  aboveskip=\medskipamount,
  belowskip=\smallskipamount,
  numberstyle=\tiny,
  %stepnumber=2,
  numbersep=10pt,
  tabsize=2,
  extendedchars=true,
  breaklines=false,
  keywordstyle=\color{black}
  }
\newcommand{\lstcd}[1]{\lstinline[basicstyle=\ttfamily\small]{#1}}

\usepackage{hyperref}
\usepackage[capitalise]{cleveref}
  % i.e., you use '\cref{x}' rather than "Figure \ref{x}"
% inlined bib file
\usepackage{filecontents}

\newcommand{\Todo}[1]{\textbf{\textcolor{red}{[#1]}}}
\newcommand{\paragraphsection}[1]{\vspace{7pt}\noindent{\textit{#1}}}
  % use this for "Initial Results" or for sections at the lowest level.
  % TODO: is the formatting sufficiently smaller than level 2 sections?
 
\sisetup{group-separator={,}, group-minimum-digits=4}

\begin{document}

\date{}

% make title bold and 14 pt font (Latex default is non-bold, 16 pt)
\title{Strengthening Weak Links in the PDF Trust Chain}

\author{
    \IEEEauthorblockN{ Mark Tullsen, William Harris}%
    \IEEEauthorblockA{\small Galois, Inc.\\
    \texttt{\{tullsen,wrharris\}@galois.com}} \and
    \IEEEauthorblockN{Peter Wyatt}
    \IEEEauthorblockA{\small PDF Association\\
    \texttt{peter.wyatt@pdf.org}}
}

\maketitle

%-------------------------------------------------------------------------------
\begin{abstract}
%-------------------------------------------------------------------------------

\Todo{..}
  
\end{abstract}


\section{Introduction}

%\subsection{The problem of secure data formats}

... Further complicating the story is the emergence of \emph{de facto standards}: ``dominant implementations of these formats extend the [published] standards by deliberately accepting non-compliant inputs without any indication to the users that the document contains malformations silently presumed benign''\footnote{SafeDocs Broad Agency Announcement.}: what now \emph{is} the de facto standard?  And how benign is it?

\subsection{Blah}

\subsection{Summary}

% In \cref{sec:learning} we discuss some of the approaches we are taking to un% derstanding and learning de facto formats:
% \begin{itemize}
%     \item Labeling and categorizing data sets used for grammar inference usi% ng extant parsers.  Corpora of samples of a given data format tend to includ% e both positive and negative examples of the format.  These examples often d% o not come with labels indicating how strictly they adhere to the target lan% guage.  By leveraging extant parsers, we generate sets of labels for each ex% ample in a corpora to build a feature space allowing this classification to % be performed.
%     \item Detection of sublanguages within formats. Many formats make use of%  sublanguages to improve their expressive power; this power naturally comes % at the cost of increased complexity of the format at large. In solving the p% roblem of inferring the structure of an entire format, it is worthwhile to d% etect, isolate, and possibly operate on strings in these sublanguages.
%     \item Grammar inference via reinforcement learning. Modern grammar infer% ence tools are predominately focused on identifying structures in natural la% nguages \cite{dyer2016rnng,kim2019compound,drozdov2019latentTreeInduction,co% hen2010jdageem} or on data formats more simple than PDF \cite{zhu2010increme% ntal,fisher2008dirt,fisher2011pads}.  We consider a novel algorithm for infe% rring data structures in {\em Visibly Pushdown Languages} (VPLs) \cite{alur2f% 004vpl} by combining reinforcement learning with bottom-up merge parsing.
% \end{itemize}


\section{Summary}


\subsection{Our Contributions}
\subsection{Related Work}
\subsection{Future Work}
\Todo{fix:}
\begin{enumerate}
    \item UI for sublanguage segments identified
    \item Roll sublanguage segments as tokens/base actions into grammar inference algorithm.
    \item Tie it all together - build a suite of UI tools for investigating folders containing data samples from a single grammar.
    \item Inferring a proper grammar from the learned parser.
    \item Demonstrate that format inference tools ease creation of lenses and other transformations for safe subsettings.
\end{enumerate}


\section*{Acknowledgements}

This research was supported by the SafeDocs program under \Todo{HR0011-19-C-0073}.


\bibliographystyle{plain}
\bibliography{old}
% \bibliography{new}

\end{document}
