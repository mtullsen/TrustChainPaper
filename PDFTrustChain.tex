\documentclass[conference,10pt]{IEEEtran}
% Re-allow e.g. the \thanks command.
\IEEEoverridecommandlockouts

% to be able to draw some self-contained figs
\usepackage{tikz}
\usepackage{amsmath}
\usepackage{minted}
\usepackage[utf8]{inputenc}
% \usepackage{todo}
\usepackage{csquotes}
\usepackage{amsthm}
\usepackage{siunitx}
\usepackage[caption=false]{subfig}
\usepackage{listings}
\lstset{
  basicstyle=\scriptsize\ttfamily,
  mathescape=true,
  %numbers=left,
  aboveskip=\medskipamount,
  belowskip=\smallskipamount,
  numberstyle=\tiny,
  %stepnumber=2,
  numbersep=10pt,
  tabsize=2,
  extendedchars=true,
  breaklines=false,
  keywordstyle=\color{black}
  }
\newcommand{\lstcd}[1]{\lstinline[basicstyle=\ttfamily\small]{#1}}

% this style is used for long/verbatim notes (not text) in the paper
\lstdefinestyle{meta}
  {basicstyle=\tiny\ttfamily\color{magenta}}

\usepackage{hyperref}
\usepackage[capitalise]{cleveref}
  % i.e., you use '\cref{x}' rather than "Figure \ref{x}"
% inlined bib file
\usepackage{filecontents}

% SYSTEM OF ANNOTATIONS:
\newcommand{\info}[1]{\textcolor{blue}{[[#1]]}}
\newcommand{\note}[1]{\noteYes{#1}}  
\newcommand{\noteNo}[1]{}  % way to remove all notes.
\newcommand{\noteYes}[1]{\textcolor{red}{[[#1]]}}
\newcommand{\todo}[1]{\note{TODO: #1}}
\newcommand{\todoa}[1]{\todo{[\#A]: #1}}
\newcommand{\todob}[1]{\todo{[\#B]: #1}}
\newcommand{\todoc}[1]{\todo{[\#C]: #1}}
\newcommand{\old}[1]{\textcolor{orange}{[[OLD: #1]]}}


\newcommand{\paragraphsection}[1]{\vspace{7pt}\noindent{\textit{#1}}}
  % use this for "Initial Results" or for sections at the lowest level.
  % TODO: is the formatting sufficiently smaller than level 2 sections?
 
\sisetup{group-separator={,}, group-minimum-digits=4}

\begin{document}

\date{}

% make title bold and 14 pt font (Latex default is non-bold, 16 pt)
\title{Strengthening Weak Links in the PDF Trust Chain}

\author{
    \IEEEauthorblockN{ Mark Tullsen, William Harris}%
    \IEEEauthorblockA{\small Galois, Inc.\\
    \texttt{\{tullsen, wrharris\}@galois.com}} \and
    \IEEEauthorblockN{Peter Wyatt}
    \IEEEauthorblockA{\small PDF Association\\
    \texttt{peter.wyatt@pdf.org}}
}

\maketitle

% abstract:
\begin{abstract}
  % context:
  Conventional data-description languages, such as context-free
  grammars, naturally define formats in which the semantic
  interpretation of a large segment of input depends hierarchically on
  the semantic interpretations of its sub-segments.
  %
  They cannot be applied to many practical and security-critical
  formats---including the Portable Document Format (PDF)---in which
  the interpretation of a segment as a \emph{Document Object Model
    (DOM)} graph depends on a concept of reference and complex
  contextual data that binds data objects to references
  \textcolor{blue}{shouldn't this be "references to data objects"?}.
  %
  Such referential context itself must often be defined
  discontinuously and compressed, to satisfy practical constraints on
  usability and performance.
  \textcolor{blue}{Sorry - I don't understand this previous sentence. How about "The integrity of these references and their context ensures that an ambiguous DOM graph is established from a basis of trust."?}

  % result:
  This paper describes a case study of a critical instance of such a
  design, namely the construction of PDF \emph{cross-reference tables},
  in the presence of potentially multiple incremental updates and multiple complex dialects expressing these references.
  %
  Over the course of our case study, we found that the full definition of
  cross-reference data in PDF contains several subtleties that are
  mis-implemented by natural implementations, but which can
  nevertheless be formalized using parser combinators in a
  data-definition language with constructs for explicitly capturing
  and updating input streams.
  %
  Our definition has served as a foundation for implementing format
  security analyses that arise naturally while constructing DOM's,
  including the analysis of document \emph{cavities} that may store
  unexamined data used to construct file \emph{polyglots}.
  
  \todo{need to be consistent in terminology: PDF Trust Chain vs Chain of Trust. Also be good to use this phrase in the Abstract. Do we quote it?}
\end{abstract}


% \section{[Meta-Notes for Authors]}

\begin{lstlisting}[style=meta]
CONVENTION:
 - using this lstlisting[style=meta] environment to capture
   text in outline form that has not been fleshed out / turned into prose.
\end{lstlisting}

\begin{lstlisting}[style=meta]
META NOTES:  
- shoot for 12 pages
- challenge: figuring out how much detail to go into, e.g., xref
- the idiom
  - details (e.g., in PDF)
  - general principles
    - E.g., such as
      - cavities
      - trust-chain 
      - redundant-data [highlight]
        - E.g., Size, we don't want to *invisibly*
          null-out obj. nums > Size
      - file-offsets in format
      - schizophrenia / polyglot
      - limitations of informal (english) standards
   - at least 1 other example of the principle
   - ICC, etc.
TERMS (to actually use, and define when needed):
- complies with standard, compatible with standard
- Pre-DOM
- ?? for the repetitious use of ``parsing and computation''
\end{lstlisting}

% content:
% ------------------------------------------------------------------------------
\section{Introduction \note{1.5pp}}
\label{sec:intro}
% background: conventional grammars:
The task of parsing may be viewed as receiving a document in an
unstructured, or serialized, form, and building its structured
representation, equipped with an argument that the structured
representation is a faithful representation of the given document.
%
For formats defined in conventional data-description languages that
correspond closely to well-understood classes of the Chomsky hierarchy
(i.e., context-free grammars, with regular expressions as a critical
special case), trustworthy arguments fall out naturally from the
definition of the format itself.
%
This is in part due to the fact that in such formats, the structured
representation of a large segment of a message is defined purely in
terms of the structure of its segments.
%
E.g., the structure of a record consisting of multiple fields (e.g., a
name, age, and social security number) is defined in terms of the
correct structure of each of the constituent fields.

% real formats have DOMs, which involve names and references
However, these key properties regarding context-freedom of a
definition critically do not hold for many practical formats of
interest.
%
Many such formats define a \emph{Document Object Model}: i.e., the
result of parsing a document is a graph between objects of data, each
of which may store a large set of fields.
%
In such formats, it is infeasible to provide context-free definition
of well-formedness because data objects that must satisfy critical
relations may not occur contiguously in a document: related objects
may not form a tree-like hierarchy.
%
Such formats typically introduce a notion of \emph{naming} or
\emph{reference} by which objects may refer to each other.
%
A critical practical example of this design pattern---and the
motivating example of our work---is the document object model of the
\emph{Portable Document Format (PDF)}~\cite{isotc171sc2wg8ISO32000220202020};
%
PDF data objects include an object \emph{identifier}, by which other
objects may refer to them.

% further complications: people want to do other fancy tricks with
% context tables
In such formats, the types used for representing references and the
data structures used to represent context take on central importance.
%
The simplest definition of such a data structure may arguably amount
to including a single, contiguous definition of context.
%
While potentially straightforward, such a design would not support
requirements of many practical formats, including PDF.
%
In many contexts, a critical feature of formats is to update a
document incrementally by appending data: this is often best achieved
by allowing the table of references to itself by updated with new,
non-contiguous entries that effectively overwrite previous conflicting
entries.
%
Additionally, such tables of references may be compressed using
standard, but non-trivial, compression algorithms.

% potential solutions and why they fail:
Context-free grammars and weaker formalisms are not suitable to define
such formats in the full detail.
%
In the conventional setting in which a parser returns a semantic value
that is then potentially transformed by further computation, which may
itself be defined in an attribute grammar or parser client logic.
%
While feasible, the main limitation of such an approach is that
computation on semantic values must then itself effectively parse
unstructured data after computing partial contextual information;
%
such parsing logic is exactly what should be expressed declaratively
in a grammar and implemented by a generated parser.
%
Alternatively, in principle it could be feasible to express some
practical formats in fundamental grammar classes that are more
expressive than those traditionally used to define formats, such as
\emph{sensitive} grammars.
%
However, for such a general class, bounds on parsing performance are
too loose to ensure acceptable practical performance.

% our solution: very careful parser combinators:
This paper explores a third approach: parser combinators a data
description language that includes constructs for explicitly capturing
input and parsing a previously-captured input, combined using
semantic-level computation in Haskell.
%
This approach is explored within an industrial strength case study:
validating and parsing the reference table of the PDF DOM.
%
In general, using parser combinators is not new: such combinators are
available widely available in the distributions of modern industrial
strength languages~\todo{cite: Monadic parser combinators, Staged
  selective parser combinators, parsec}.
%
Moreover, with the recent interest in formalizing practical formats
and generating high-assurance parsers, such combinators have
specifically been applied specifically to formalize components of the
PDF standard related to referential context.
%
However, our work is unique in that, to our knowledge, it constitutes
the first attempt to use such combinators to formalize a comprehensive
set of features used in PDF to define referential context,
specifically incremental updates to reference tables and reference
table compression within cross-reference streams.
\textcolor{blue}{I think there is also an element of general file layout (such as cavities) as an incremental update implies it is actually incremental and thus byte offsets have some inferred semantics (cf. shadow attacks; new objects cannot be from earlier in the file or a future incremental update)} 

% results:
The results of our case study show that such features can apparently
by formalized, although the resulting definition is somewhat more
subtle than what may be implemented based on a natural interpretation
of the PDF standard, or many extant documents. \textcolor{blue}{Should we mention about here that past work on PDF has focused on the DOM, without considering how trustworthy or integral the DOM is?} 
%
However, we suspect that there is nothing in the format definition
that requires the language of combinators that we applied: a key goal
of our work is to provide this formalized definition as a worked case
study, to be improved upon using definitions in other experimental
data definition languages as they are developed.

% further applications:
Using our comprehensive formalization of referential context, we have
developed several tools for assisting format developers to determine
if their format may suffer from sophisticated threats to security that
are intricately connected to the case of parsing and constructing object models,
specifically detecting the presence of document \emph{cavities}
segments of documents that do not affect the document's semantic
value, and thus may store content that is completely unobservable to
parser clients.
%
Such cavities are a powerful mechanism for creating \emph{polyglot
  files} (i.e., files that belong to multiple, unexpected formats),
which themselves have been used in recent critical system security
exploits \todo{cite iOS vuln}.

\todo{short paragraph on PDF and its challenges}
\begin{lstlisting}[style=meta]
- Creation of DOM ("document object model")
  - list of object definitions
  - ... containing object references
  - designated root object
- Cross-reference (xref) table
  - table with byte offset for each object
- Cross-reference streams (added in PDF 1.5) [7 pp. in spec]
  - compressed, complicated, space-efficient, ...
  - allows for hybrid files with both traditional
    xref tables and new xref streams
- Incremental updates
  - by only appending to PDF file we can add, update, delete, restore objects
- Linearized PDF (efficient incremental access) a.k.a. "Fast web view"
  - "differential by design"!
\end{lstlisting}
\begin{lstlisting}[style=meta]
- We must accurately create the DOM (a DAG, no cycles allowed)
  - ... while abstracting over xref tables, xref streams,
    hybrid files, incremental updates, linearization
  - ... while recovering from errors
  - ... while doing xref table reconstruction
- This is the source of errors, ambiguities, and vulnerabilities!
\end{lstlisting}

\todo{short paragraph on PDF Vulnerabilities}
\label{sec:pdf-vulnerabilities}
\begin{lstlisting}[style=meta]
- Ambiguous files
- Polyglot files
- Shadow attacks: possible because of ability to sign dead objects
  and cavities 
  \pwnote{and, depending on attack mode, referencing "dead objects" via a later
  incremental update}
\end{lstlisting}

\todo{complete}
\paragraph*{Organization} The rest of this paper is organized as
follows;
%
\Cref{sec:background} provides technical background on which our work
is based;
%
\todo{complete}
%
\Cref{sec:rel-work} reviews related work, and %
\Cref{sec:conclusion} concludes.



% bkg: technical stuff that we didn't contribute
\section{Background}
\label{sec:background}

\subsection{The PDF Trust Chain \note{1.5pp}}
\label{sec:trust-chain}

Let's define what we mean by ``PDF Trust Chain''.

\subsection{Trust Chains}

The term \emph{Trust Chain} is used in multiple contexts, e.g.,
\emph{digital certificates}: a sequence of certificates signing certificates,
starting with a root certificate;
\emph{supply chain}: a product is no more reliable or secure as its
outsourced components;
\emph{trusted boot}: unless the bootloader is correct and non-malicious,
there can be no possibility of the operating system being the same;
\emph{software stacks}: upper layers are dependent upon lower layers (such as
system libraries) and vulnerabilities at the bottom affect all layers above.

The common idea is that we have layers/components that rely on lower
\textcolor{blue}{is "lower" the correct term? Or "prior processing" and then "subsequent processing"?}
layers/sub-components/etc for their validity.
And the key lesson being,
{\bf{if a single element of the trust chain 
  is flawed or suborned, then every element ``above'' it
  is no longer capable of being trusted.}}


\subsection{The Trust Chain of a PDF Parser}

% In \cref{sec:pdf-challenges}, we elaborated on the challenges of PDF.
% Parsing data-formats has a long history and many solutions ...
% Parsing formal languages also has a long history and many solutions ...
% PDF has aspects of both: this makes PDF challenging.
% But PDF ``parsing'' is not merely a matter of harder [difference of degree]
% but intrinsically more complex [a difference of kind!]:

\begin{figure}[t]
    \centering
    \includegraphics[width=0.8\linewidth]{figures/trustchain-diagram.png}
    \caption{The PDF Trust Chain diagrammed.}
    \todo{make diagram prettier!}
    \label{fig:pdf-trust-chain}
\end{figure}

In \cref{sec:pdf-challenges} we touched upon the complexities of parsing
PDF, but to appreciate these, one has to understand the
dependencies and interactions between the features.
In \cref{fig:pdf-trust-chain} we show the main components diagrammatically.
To briefly sketch what's going on here:
\begin{itemize}
\item Phase 1: we find the PDF header \verb|%PDF-x.y| (near start of the physical file, to account for preamble), then the end of the PDF file \verb|%%EOF| (near the end of the physical file), then "backwards parse" to find the last \verb|startxref| keyword followed by an end-of-line sequence and an integer value encoded as a sequence of ASCII bytes representing the byte offset in the PDF file (which is then adjusted for any preamble to a physical byte offset), and then locate either the \verb|xref| keyword for traditional PDF cross-reference tables, or a PDF object that should be a cross-reference stream. In the case of traditional PDF cross-reference tables, after the cross reference table will be the trailer dictionary identified by the \verb|trailer| keyword or, alternatively for PDF 1.5 and later files with cross-reference streams, the trailer dictionary keys will be in the stream extent dictionary of the cross reference stream. Of particular note is the \verb|Size| entry, which is one greater than the largest object number allocated in the PDF file. 
\item Phase 2: using information from Phase 1, we find and parse any incremental updates. These are identified by a \verb|Prev| entry in either the trailer dictionary or the stream extent dictionary of a cross-reference stream. The value of the \verb|Prev| key is another byte offset to the immediately preceding incremental update which, again, can either be a traditional cross-reference table and to the start of the \verb|xref| keyword, or to a cross-reference stream. This process  repeats, working from the most recent update back through time to the original PDF document.
\item Phase 3: data in each cross reference table must then be parsed to identify the byte offset to the start of each PDF object. Note also that PDF does not define the byte offset to the end of an object. There are two sets of objects in every PDF document: the in-use list of PDF objects and a free list of PDF objects. Object zero is always the start of the free list as it is not otherwise a valid object number. For incremental updates, PDF object numbering does not have to be sequential, with skipped object numbers assumed to be on the free list (although this is not stated explicitly in the PDF specification). Parsing depends on the form of the incremental update, with traditional cross-reference tables being simpler and larger independent of other processing. Cross-reference streams however are more complex as they are usually compressed and thus require the pre-DOM parser to "trust" the stream extent dictionary data.
\item Phase 4: using information from Phases 1, 2 and 3, the final set of objects that comprise the final PDF document can be established. Each incremental update can add new objects, mark existing in-use objects as free, or reinstate previously freed objects. \textcolor{blue}{do we want to mention the complexity of dig-sig here? what about cavities?} 
  \todo{More phases too?}
\item Phase 5: \todo{...} The result is the candidate DOM, 
  the candidate DOM is a mapping \lstcd{ObjId -> PdfValue}.
\item Phase 6: This phase takes the candidate DOM and verifies that
  it represents a sensible Document Tree per the PDF Standard.  E.g.,
  all indirect objects are in the candidate DOM, no unexpected recursion,
  etc.
\item Render Phase: we can now render the validated DOM, or parts thereof, to
  whatever display or graphic format we choose.
\end{itemize}

Note that phases 2, 3, 4 and 5
all require inputs from the previous phase to enable them to know where and
how to parse further segments of the PDF input file.
%
Note this: an implementation \emph{might} merge phases 1-3 into single phase
and give a \emph{semblance} of simplicity, but our argument (in what
follows in \cref{sec:specifying}) is that such an implementation will be
overly complex and we will have a near insurmountable task to assure that it
terminates for all input files.

% Only after Phase 4, have we correctly constructed a 'candidate' DOM (Document
% Object Model) where we have a mapping of Object Identifiers to PDF Values.

Although verifying Phase 5 is both difficult and tedious
(\todo{...; say something re the Arlington DOM model? Paper?})
and the Render phase has many complications of its own (fonts are
a special challenge, etc.), in this paper we focus on Phases 1-4.
We call these phases the pre-DOM parsing/computation, and if anything
goes wrong pre-DOM, lots can go wrong, the wrong DOM may be rendered, etc.!

\todo{Need to mention that validating a PDF with a digital signature entails identifying at which iteration of the PDF document the dig-sig was applied and then validating the dig-sig in the context of that specific DOM and the objects that were in effect at that instance in time.}

The attentive reader will note that we have another instance of a \emph{Trust
Chain}.  The later phases of the parsing process are \emph{completely
dependent} upon the earlier phases to properly parse and interpret the PDF
file.

% The PDF "trust chain": higher levels of abstraction depend upon lower levels.
% These structures are not necessarily concrete values--e.g. parsed xref
% table--but they do exist `conceptually'.

\todo{where do we want to mention ICC processing which has a simpler "Chain of Trust"? Should the text below not be PDF specific, but generic to file formats}

We think it is important to understand PDF parsing in terms of this
\emph{Trust Chain} as
%
(1) it highlights the presence of the many ``dependent'' parsers (or phases)
in PDF processing.
%
(2) it highlights the importance of ensuring the pre-DOM parsing, data integrity relationships and
computation (the base of our Trust Chain) is correct and secure.
%
(3) it reminds us that the integrity of the DOM cannot be verified
independently of the lower levels.

\todo{say ... regarding our repetitious use of ``parse and compute''}
\todo{we should use the concept of "data integrity relationships" being the essence of the necessary context. e.g. PDF incremental updates are appended to a previously valid PDF - thus an incremental update should not "make visible" a PDF object that was not already valid and visible in the original PDF (this is one method Shadow Attacks use - cf. an upstream "supply chain attack" by an attacker), even if that PDF object is otherwise entirely syntactically valid. In the same way an incremental update that refers to an object at a file byte offset after the incremental update would be highly suspicious.} 
\todo{ensure that spec.hs, the above text, and \cref{sec:specifying}
      are all consistent!}

\subsection{Root Causes of PDF Complexity}

Most data formats can be described by much simpler mechanisms;
most language processors (e.g., a Python parser) can be described and parsed by
textbook methods (e.g., the old \emph{lex} and \emph{yacc} are sufficient for
most language processors);
so what makes PDF processing so much more complex?
\begin{lstlisting}[style=meta]
  - indirect offsets
    - which may recursively point to other indirect offsets
    - need programming language
      (or a 'seek' in the data definition lang)
  - DOM is a directed graph structure that allows cycles and arbitrary references (so not a DAG)
    - objects point to objects via byte offsets (cf. not by nested expressions such as XML)  
  - backwards parsing
  - some parsing rules written as to be not relevant to parsers ("writer-only rules")
  - incremental updates as a set of deltas to be applied to the file, which change the DOM (note: unlike HTML, PDF's DOM is fixed and cannot be altered by JS)
  - xref tables ...
  - ... giving rise to cavities
    - giving rise to polyglots
  - dependent parsers
  - <MORE?>
\end{lstlisting}

Further detail of how these work in PDF is in \cref{sec:specifying}.

% ------------------------------------------------------------------------------
\section{Pre-DOM Vulnerabilities \note{2.5pp}}
\label{sec:predom-vulnerabilities}

As will become even more apparent, there is a significant amount of
parsing and computation that needs to be done \emph{pre-DOM}.
And given our recent points about the \emph{PDF Trust Chain}
(\cref{sec:trust-chain}),
it should not surprise us that most of the PDF attack vectors
(\cref{sec:pdf-vulnerabilities})
involve some aspect of breaking the \emph{DOM} abstraction.
I.e., they occur at the \emph{pre-DOM} levels.

{\bf{Shadow Attacks}} \todo{...}

{\bf{Schizophrenia}} \todo{should define what we mean by schizo - is this the same thing that gives rise to parser differentials or a feature of of the syntax or layout of PDF?}
\begin{lstlisting}[style=meta]
  - writer errors
  - parser differentials
    - e.g., ignoring xref tables
  - recovering parsers !!
  - blind faith in incremental updates (Shadow Attacks)
\end{lstlisting}

{\bf{Polyglots}} 
\todo{... arising from cavities and permissive implementations and ...}

{\bf{Denial of Service (DOS)}} 
%
\begin{lstlisting}[style=meta]
- [potential recursion many places]
- format may not be well-defined because the recursion is not
    "well-defined"
\end{lstlisting}

{\bf{Steganography}} \todo{?}

{\bf{Others}} \todo{have any others here?? Maybe PII/redaction issues - just 'cos you delete something in a PDF doesn't mean it is really deleted (thanks to incremental updates)!!!! }


% A literate Haskell file

\section{Specifying the DOM's Foundations \note{4pp}}
\label{sec:specifying}

\subsection{Specifications}
% REMEMBER: [terms: complies with standard, compatible with]

What do we want from a PDF implementation?
Among other things we want it to
\begin{itemize}
\item comply with the standard,
\item when not supporting full standard, do so gracefully,
\item support common extant data malformations
  to be compatible with extant data, without introducing ambiguities
  or causing other unintended consequences.
\end{itemize}
At the same time, it should be resilient against all attacks.

The job of an implementor is challenging due to many factors:
\begin{itemize}
\item The intrinsic complexity of PDF:
  PDF is a \emph{less than ideal} design that reflects 27 years of
  an evolving standard.
  PDF has multiple redundant features.
  PDF is architected to allow for efficient implementations of
  large files (``efficiency hacks'').
  PDF involves multiple sublanguages and embedded formats.
  PDF contains complex parsing needs:
  E.g., searching backwards, embedded file offsets.
\item Lack of formality in standard. Thus, implementations
  require more effort to comply.
  Implementations commonly over implement, under implement,
  and wrongly implement the standard.
  Clearly, writing a PDF implementation from scratch is challenge,
  implementors are incentivized to patch existing code (correct or not).
\item The standard defines one thing: \emph{What is a valid PDF?},
  and leaves other decisions to the implementation:
  (1) What should absolutely \emph{not} be allowed (because in the real world
    implementations are more or less relaxed)? And similarly,
    what are deemed to be acceptable, reasonable error recovery methods?
  (2) What is required to support backwards and forwards compatibility?
  (3) What is done when redundant features are inconsistent: which, or
    neither, has priority?
    Similarly, what is done when the stated PDF version and the PDF
    constructs used don't match?
  (4) What is \emph{required} from the PDF writer versus
    what do we \emph{require} the PDF reader to check?
  (5) If given an incorrect PDF, is the behavior undefined\footnote{
      I.e., like ``undefined'' in the C programming language sense: the
      compiler can do anything it wants!
    } or \emph{must} we detect and report this?
\end{itemize}
\mtnote{this last, (5), seems pretty key and
  feels like a major omission in the standard! If the answer is
  yes, then there's nothing keeping an implementation from only
  doing enough to ``work on good pdfs'' without even checking that it
  IS a good PDF.  Thus, we have the tool that only reads the 'e'
  in ``endobj''.  Is this reasonable?  Are there vulnerabilities that
  would exist in such implementations?
  }
Failure to faithfully implement the standard can result in ambiguities
between implementations as well as direct vulnerabilities (such as
Shadow Attacks or DOS attacks).

Writing a \emph{formal specification} is not going to solve all the above,
but we believe it is a strong first step toward clarify the
standard, understanding the vulnerabilties, and aiding the PDF implementor.

We use Haskell \cite{Haskell} as a specification language for
our specification.
%
The scope of our specification is the gap between the low
level parsers (parsing integers, parsing XRef entries, etc.) and the
processing that happens after the DOM is created (stages 5 \& 6).
%
Thus, the primitive parsers are assumed, not included in this spec,
other formalisms\footnote{Such as daedalus, see \todo{}
} would be better suited to specifying the parsers.
\mtnote{need?: Also, for the sake of focusing on the tricky parts and ignoring
  the tedious (not that it is always obvious which is which),
  we do not show code for many of the simple, tedious parts.
}
See \cref{sec:appendix1} for the Haskell type signatures for the
parsing primitives and various other functions.
%
The full spec can be found online at \cite{daedalusrepo}.

\begin{itemize}
\item Our specification is formal and executable\footnote{
  Executable does not imply efficient, the specification is written
  primarily to be \emph{clear}.}.
  %
  This is our motivation for choosing Haskell over pseudo-code,
  English prose, and non-executable formalisms.  For the reader
  without a reading knowledge of Haskell, we understand that parts of
  the specification could be a little obscure, but our hope is that a
  precise, formal specification may prove to be more useful than
  pseudo-code or the like!
  
\item Our specification is purely functional: no ad hoc global variables are
  hidden, the data-dependencies in the spec fully represent all the
  data-dependencies.  Motivated by the Trust Chain issues
  (\cref{sec:trust-chain}), our objective was to capture all dependencies.
  
\item Our specification hides no difficulties: one could implement a PDF parser
  by writing the omitted functions, but the spec stands complete, as
  is\footnote{One caveat: the addition of support for \emph{some} features
  could require some re-design.  E.g., signature validation, as discussed in
  \cref{sec:updates-and-signatures} would require non-trivial changes.
  }!
\end{itemize}

% need to say: These structures are not necessarily concrete values--e.g. parsed
% XRef table--but they do exist `conceptually'.

This spec supports PDF 2.0, including compressed objects and XRef streams.
%
It (safely) ignores linearization data, and in hybrid XRef PDFs
it ignores the traditional xrefs for pre 1.5 readers.
It processes signatures (as incremental updates) but it does not support
validation of signatures.

\mtnote{brings up issues:
  How much validation to do?
  Are we specifying a renderer or a text-extractor?
  Are we supporting the validation of signatures (which radically
  changes things)?}

\subsection{Stages 1 - 4, Preliminaries}

%%%% begin: Hs code not in paper %%%%
\iffalse
\begin{code}
{-# LANGUAGE EmptyDataDecls, TypeOperators, LambdaCase #-}
module Spec where
import           Control.Monad
import           Data.Char
import           Data.Foldable(foldlM)
import qualified Data.Map as M
import           Data.Map(Map)
import           Types
import           Utils
import           Primitives
import           Streams
\end{code}
\fi
%%%% end: Hs code not in paper %%%%

What follows is the definition of \lstcd{pPDFDOM} which does the
parsing \& computation for stages 1-4;
it creates the \lstcd{DOM} relying on primitive parsers.
\begin{code}
pPDFDom :: P DOM
\end{code}
The above line is a type signature, it is saying that
\lstcd{pPDFDom} is a ``monadic parser'' \lstcd{P} that returns a
value of type \lstcd{DOM}.
In Haskell a monad can sequence many effectful constructs: global variables,
and etc. However, \lstcd{P} is a simple monad that effectively has one
read-only variable (the PDF file being read), and one mutable variable,
the offset in the file where reading (parsing) occurs.  I.e., we have
a single effectful primitive monadic function\footnote{
  We'll use the term ``action'' to refer to a monadic function:
  i.e., a function having a type scheme of ``\lstcd{a -> P b}''.
}:
\begin{codeNoExecute}
seekPrimitive :: Offset -> P () -- move read pointer to offset in file.
\end{codeNoExecute}

\subsection{Stage 1: Find \& Parse Header and Trailer}

Now to define \lstcd{pPDFDom}:
\begin{code}
pPDFDom =
    do
    -- find '%PDF-x.y' at start of file, searching the first 1000 bytes:
    (version, headerOffset) <- findPDFHeader
    -- search backwards from EOF for 'startxref', gives up after 1000 bytes:
    (startxrefOff, xrefOff) <- findStartxrefThenParseToEOF
\end{code}

\haskellnote{The \lstcd{do} keyword indicates the start of monadic code.}
The action \lstcd{findPDFHeader} has type \lstcd{P ((Int,Int),Offset)};
it can fail if the header cannot be found or is malformed.
%
Note also that the above two function calls have no data dependencies, they
could be done in either order.

A non-obvious feature of PDF is that file offsets are in relation to, not the
beginning of the file, but the beginning of the \emph{PDF header}.
%
The next line abstracts over this problem once and for all:
\begin{code}  
    let seek n = seekPrimitive (headerOffset+n)
\end{code}
We will need to pass \lstcd{seek} to any actions that need to change
the offset in the file.

The code for \lstcd{findStartxrefThenParseToEOF} is
not elucidated here as it is more tedious than instructive.
In English,
\begin{quote}
Find the ``EOF marker'' \lstcd{\%\%EOF} (near the end of the physical
file), then "Backwards parse" to find the last \lstcd{startxref}
keyword followed by an end-of-line sequence and an integer value
encoded as a sequence of ASCII bytes representing the byte offset in
the PDF file (which is then adjusted for any preamble to a physical
byte offset), and then locate either the \lstcd{xref} keyword for
traditional PDF cross-reference tables, or a PDF object that should be
a cross-reference stream.  In the case of traditional PDF
cross-reference tables, after the cross reference table will be the
trailer dictionary identified by the \lstcd{trailer} keyword or,
alternatively for PDF 1.5 and later files with cross-reference
streams, the trailer dictionary keys will be in the stream extent
dictionary of the cross reference stream.
\end{quote}
% TODO: reduce indentation

\subsection{Stage 2: Find \& Parse Incremental Updates}

\lstset{numbers=right}
\begin{code}
    seek xrefOff
    (xrefRaw, xrefEndOff) <- pXrefRaw :: P (XRefRaw,Offset)
    validate $
      verifyXrefRaw xrefRaw
        -- - this ensures no duplicate objectIds
        -- - we might parse and validate XRef entries at this point
    seek xrefEndOff
       -- This seek is needed because pXrefRaw doesn't need to read
       -- the contents of each XRef subsection, so let's leave the
       -- current file read location after the end.
\end{code}
\lstset{numbers=none}

For the sake of lucidity, we sometimes add \emph{type annotations} in
the code, note in line 2 of the above that \lstcd{::P(XRefRaw,Offset)}
indicates the type of the expression preceding it.
%
Note \lstcd{validate} in line 3, this is a special function that we apply to
semantic checks (line 4) that are not necessary to create the DOM but which
could detect invalid or inconsistent PDFs.

For this initial XRef table---even without parsing XRef subsections---we
know the list of object ids and we can find the XRef entry for each object id.

\begin{lstlisting}[style=meta]
 - [maybe some of this covered in previous section]
 - enforcing full standard compliance with 20 byte (only) XRef entries.
    - currently 19,21 byte XRef entries are considered NCBUR!
 - if we were to allow 19-21 byte XRef entries, we'd need
   to parse a lot more strictly and sooner.
 - nothing essential would change in our spec
\end{lstlisting}

\mttodo{this is traditional XRef specific, update!}
  
Now to parsing more of the PDF trailer
\begin{code}
    pSimpleWhiteSpace -- no comments allowed between XRef table and 
    keyword "trailer"
    trailerDict <- pDictionary
    validateAction $
      -- ensures nothing in the cavity between dictionary and ``startxref''
      do
      cs <- readToPrimitive startxrefOff -- get bytes up to `startxrefOff`
      return (all isSpace cs)

    trailerDict' <- dictToTrailerDict trailerDict
    let mPrev = trailerDict_getPrev trailerDict' :: Maybe Offset
        etc = trailerDict_Extras trailerDict'    :: Dict
          -- etc is a list of unknown key-value pairs
\end{code}

Note \lstcd{validateAction}: it differs from \lstcd{validate} in that
its argument can have side-effects (fail or change the file reading point).

\begin{lstlisting}[style=meta]
 - generally we aren't checking dictionary keys, but ...
 - pSimpleWhiteSpace - comments aren't allowed
 - if we don't do 'validate', we have cavities!
\end{lstlisting}

\mttodo{we have now parsed the trailer but ...}

\begin{code}
    updates' <- pAllUpdates mPrev :: P [(XRefRaw, TrailerDict)]
       -- we've followed the 'Prev's and for each we
       --   - pXrefRaw     -- parse XRef subsections (at raw level)
       --   - pTrailerDict -- similar to above, but
       --                     only reads/validates Prev key
    let updates = (xrefRaw,trailerDict) : updates'
\end{code}

\mttodo{integrate this text:}
... these are identified by a \lstcd{Prev} entry in either the trailer
dictionary or the stream extent dictionary of a cross-reference stream. The
value of the \lstcd{Prev} key is another byte offset to the immediately
preceding incremental update which, again, can either be a traditional
cross-reference table and to the start of the \lstcd{xref} keyword, or to a
cross-reference stream. This process repeats, working from the most recent
update back through time to the original PDF document.

\begin{lstlisting}[style=meta]
at this point
 - we know ALL the object ids in PDF
 - we have
   - parsed/validated minimally
   - rejected *some* invalid PDFs
   - no PDF 'values' parsed except trailer dictionaries
 - we can (without further 'parsing' or reading of input)
   - output trailer dictionaries
   - output high level info wrt incremental updates
 - we've detected
   - overlapping ObjIds in an XRef table (and ...?)
 - we have NOT
   - parsed anything inessential to creating DOM
   - parsed the contents of XRef entries
\end{lstlisting}

Now we finish the definition of \lstcd{pDOM}, making calls
to the actions for stage 3 and stage 4.

\begin{code}  
    -- Stage 3: combine all the updates to get a single map to offsets
    xrefs <- combineAllXrefTables updates
             :: P (Map ObjId (Offset :+: Type2Ref))

    -- Stage 4:
    dom <- transformXRefMapToObjectMap seek xrefs
    
    -- Miscellanea:
    version' <- updateVersionFromCatalogDict dom version
    if version' > (2,0) then
      warn "PDF file has version greater than 2.0"
    else
      -- version' <= (2,0)
      when (not (null etc)) $
        warn "trailer dictionary has unknown keys (per PDF 2.0)"
    validate $
      versionAndDomConsistent version' dom
    return dom
\end{code}

Due to PDF-1.5 additions, we are now in the odd position that we
cannot determine the PDF version until after we have created the DOM.
So although we
can check that the created \lstcd{dom} is consistent with
\lstcd{version'},
% 
we cannot use \lstcd{version'} to validate any of the other processing
in stages 1-4.
%
For instance, one would like to verify that Object Streams are not
used when the version is PDF-1.4 or earlier.
%
Such a check would be possible if we were to update the spec to pass
more information through the stages so we could verify more after
computation of the DOM.

In the next two sections we will look at the definitions of
\lstcd{combineAllXrefTables} (stage 3) and
\lstcd{transformXRefMapToObjectMap} (stage 4).

\subsection{Stage 3: Merge Incremental Updates}

Of particular note is the \lstcd{Size} entry, which
is one greater than the largest object number allocated in the PDF
file.
\mttodo{for last; either: implement enough to expose this or add verbiage}

\mttodo{integrate PW's text:}
In this stage, data in each cross reference table must then be parsed to
identify the byte offset to the start of each PDF object. Note also that PDF
does not define the byte offset to the end of an object. There are two sets of
objects in every PDF document: the in-use list of PDF objects and a free list
of PDF objects. Object zero is always the start of the free list as it is not
otherwise a valid object number. For incremental updates, PDF object numbering
does not have to be sequential, with skipped object numbers assumed to be on
the free list (although this is not stated explicitly in the PDF
specification). Parsing depends on the form of the incremental update, with
traditional cross-reference tables being simpler and larger independent of
other processing. Cross-reference streams however are more complex as they are
usually compressed and thus require the pre-DOM parser to "trust" the stream
extent dictionary data.

Each incremental update can add new objects, mark existing in-use objects as
free, or reinstate previously freed objects.
\mttodo{address the above in spec: either add to spec OR indicate that
  we are not ``refining'' this.}

\begin{code}
-- | combineAllXrefTables updates - 
--   - for each update
--     - parses each XRef subsection into a list of XRef entries
--   - merges all the XRef tables into a single mapping
--     - when no errors/inconsistencies
combineAllXrefTables
  :: [(XRefRaw, TrailerDict)] -> P (Map ObjId (Offset :+: Type2Ref))
combineAllXrefTables updates =
  do
  updates' <- mapM pUpdate updates  
  indices' <- mapM (createIndex . fst) updates' 
  index    <- foldlM mergeIndices M.empty indices'
  return index
\end{code}

\begin{lstlisting}[style=meta]
 - we've lost information:
   - which update an object is part of
   - object history
   - object definitions that are no longer reachable
 - fails on
   - malformed XRef entries
   - mixture of XRef table and XRef streams [PW?]
 - should detect (or fail) on
   - trailer dicts that aren't consistent between updates
   - incremental updates that are "weird/nonsensical"
     - free-ing dead objects
     - unconventional use of generation numbers
 - IF updates are defined by XRef STREAMS
   - no problem: as we can fully parse XRef stream (w/ dict) as
     there is no dependence of XRef STREAMS on DOM
     - NOTE: clarificaton to PDF working group regarding this.
   - we'll have Type2Ref's in addition to Offset's
      
 - NOTE 
   - when the latter, the ObjectId -> Offset must be available
       - in current or previous (or next!) XRef stream
         - BTW, pervasive design issue: must partial updates be valid?
\end{lstlisting}

\pwnote{and, depending on attack mode, referencing "dead objects" via a later
incremental update}

\subsection{Stage 4: Transform XRef Map to Object Map (DOM)}

Getting this stage right was a key motivation for writing our
specification, and it turned out to be surprisingly complicated.
\begin{code}
transformXRefMapToObjectMap
  :: (Offset -> P ()) -> Map ObjId (Offset :+: Type2Ref) -> P DOM
transformXRefMapToObjectMap seek xrefs0 = do
\end{code}
It has three separate stages all of which do some reading/parsing of
the file. Stage 4.1
transforms \lstcd{xrefs0} into \lstcd{xrefs1}, note the types:
\begin{codeNoExecute}
  xrefs0 :: Map ObjId (Offset               :+: Type2Ref) 
  xrefs1 :: Map ObjId (TopLevelDef_UnDecStm :+: Type2Ref)
\end{codeNoExecute}
\haskellnote{
We are using the infix type operator \lstcd{a :+: b} as a synonym for
\lstcd{Either a b}, \lstcd{Either} is the Haskell sum type having two constructors
\lstcd{Left} and \lstcd{Right}.
}
%
We look up traditional \lstcd{XRef}'s and parses the objects,
see line 3
\begin{code}
    -- Stage 4.1: parse all uncompressed objects
    xrefs1 <- mapM
                (mMapLeft (\o-> do {seek o; pTopLevelDef_UnDecStm}))
                xrefs0
\end{code}
We can't do more in stage 4.1 because we can't yet decode
object streams nor can we decode streams: in both cases these top
level objects \emph{cannot} always be parsed without looking up
integers in the DOM: as \lstcd{Length} keys and similar might contain
indirect references to top level integers.
\keypoint{
  In PDF we need to lookup and parse some objects in the DOM
  before we can parse other objects in the DOM!}
% TODO: ^ rewrite to make clearer.

Stage 4.2 further refines the xref map, computing the second from the first
\begin{codeNoExecute}
  xrefs1 :: Map ObjId (TopLevelDef_UnDecStm :+: Type2Ref)
  xrefs2 :: Map ObjId (TopLevelDef          :+: Type2Ref) 
\end{codeNoExecute}
in this code
\begin{code}
    -- Stage 4.2: decode stream bytes, pre-process ObjStm streams
    xrefs2 <- mapM
                (mMapLeft (extractStreamData xrefs1))
                xrefs1
\end{code}
We have a sufficiently complete \lstcd{DOM} in \lstcd{xrefs1} in which
we can lookup the integers that are needed to finish decoding the
streams we left undecoded.
If we cannot find the referenced integer in the DOM, we have a true PDF
error.  Any indirect \lstcd{Length} values are required to be at the
top level.

At this point, in \lstcd{xrefs2}, we still have \lstcd{ObjId}'s that point
(via \lstcd{Type2Ref}) into \lstcd{ObjStm} streams.
Only in stage 4.2 were we able to decode the stream inside of
which is the object to be parsed.
% this text is rather complicated/tedious!

\mttodo{bring in some of the Alg. Data Type defs?}

So at this point, (1) we can compute body cavities \todo{define this} as
we can parse all the top level objects, (2) object streams have been
pre-processed, but we have not yet parsed the objects inside them.

Finally, stage 4.3 computes the second from the first
\begin{codeNoExecute}
  xrefs2       :: Map ObjId (TopLevelDef :+: Type2Ref) 
  domCandidate :: Map ObjId TopLevelDef
\end{codeNoExecute}
using the same pattern we've been seeing but now we no longer have 
the sum, we have a mapping from \lstcd{ObjId}'s to PDF Values:
\begin{code}
    -- Stage 4.3: resolve Type 2 references
    domCandidate <- mapM
                     (return `either` derefType2Ref xrefs2)
                     xrefs2
    return domCandidate
\end{code}

At this point every object referenced via the XRef table has been
correctly parsed. However, (1) object definitions in the PDF body
area that are not ``XRef referenced'' are neither read nor parsed,
% 
(2) objects inside an \lstcd{ObjStm} stream that are not ``XRef
referenced'' are also neither read nor parsed.

Note that the \emph{Catalog Dictionary} (which has the optional
\lstcd{Version} key) may have been in an Object stream, so in general
we may not know till this point which version of PDF we are
parsing\footnote{We wonder if this was the intention of the Standard!}.

\subsection{A One Stage Version?}
\label{sec:single-pass-problems}

\newcommand{\ssp}{$S$}
\newcommand{\dsp}{$D$}

One might ask if one might implement the above as a single stage?
Or another question, can we rewrite the specification to be a single stage?
The answer is \emph{Yes, but at a cost we don't want to pay!}
Let's refer to the above multi-stage specification as the \ssp{}
(staged) specification.
%
We will compare it to a single-stage specification, the \dsp{}
(dynamic) spec in which stage 1 is the same but stages 2, 3, and 4 are
merged together.
%
In our experience, PDF tools generally work more like \dsp{} than \ssp{}.

\dsp must be understood imperatively; here's a sketch of a possible
implementation in C-ish notation: \mttodo{put into figure; fixup lst rendering.}
\begin{myc}
  // xref and dom start with nothing in them.
  // - their size would in reality be dynamically allocated.
  XREF_ENTRY xref[Size];     
  DOM_ENTRY dom[Size];
  
  // the DOM is updated dynamically, on demand, via the following
  PdfValue deref(ObjId oi) {
    if (evald(dom[oi]))
      return dom[oi];
    else if (infiniteloopdetected())
      quit ();      
    else {
      o = getOffsetFromXref(xref, oi); // updates xref[oi]
      seek(o);
      v = parseObject;
      dom[oi] = v;
      return v;
    }
    
  parseObject {
    /* ... */ ; len = deref(oi'); /* ... */
  }
  
  OFFSET getOffsetFromXref(xref,oi){
    if xref_evald(xref[oi]) then
      return xref[oi];
    else {
      // follow Prev pointers if not in top xref
      //  - make sure no infinite loop in chasing Prev's
      /* ... */
      xref[oi] = offset;
      return xref[oi];
    }
  }
\end{myc}
The key things to note about the pseudo-code are
\begin{itemize}
\item \lstcd{XREF_ENTRY} and \lstcd{DOM_ENTRY} are both types that mutate progressively
   from unparsed/unknown to fully evaluated.
\item \lstcd{deref()} and \lstcd{parseObject()} are mutually recursive functions.
\item implementing \lstcd{infiniteLoopDetected()} is non-trivial.
\end{itemize}
Also to note about \dsp{}:
\begin{itemize}
\item it is \emph{not} equivalent to \ssp{}: \dsp{} potentially reads
  less of the input file and accepts more input files.  it is
  naturally ``lax'': it would for instance allow a \lstcd{Length} to
  be stored in an \lstcd{ObjStm} stream as long as an infinite loop
  wasn't detected.  However, we could extend (and complicate) \dsp{}
  to approximate \ssp{} better.
  % E.g.,
  %  - have a =derefLength= / =derefFromUncompressed=
  %  - More complicated than just this, because this won't catch error if we
  %    luck out and when we request the length it is already decoded.
\item it is nicely lazy if the PDF tool doesn't need to \lstcd{deref}
  every object identifiers, even more lazy than \ssp{}.
\end{itemize}

We believe that \ssp{} is clearly preferable to \dsp{} for these two
simple reasons:
\begin{itemize}
\item It corresponds to the standard, and does so obviously.
\item It does not use general recursion, so it is obvious that this
  algorithm terminates on all inputs.
\end{itemize}
While \ssp{} also gives us a few other advantages over \dsp{}:
\begin{enumerate}
\item The functional, declarative, and typed structure enables
  us to understand the stages conceptually.  (Even if one chooses
  not to implement in a like manner.)
\item It demonstrates the non obvious fact that one can
  implement stages 2-4 and keep the ``state of evaluation'' of each
  object identifier the same in each pass.
\item We know exactly what is and isn't parsed, regardless of the
  order in which one traverses the \lstcd{xref}.
\item It is intrisically parallelizable due to the extensive use of
  map-like combinators.
\item The declarative nature of \ssp{} makes it very amenable to
  modification: e.g., the simple addition of the \lstcd{validate}
  operator.
\end{enumerate}

\subsection{Incremental Updates and Signatures: What A Tangled Web \note{0.3pp}}
\label{sec:updates-and-signatures}

\todo{validating a PDF with a digital signature entails
  identifying at which iteration of the PDF document the dig-sig was applied and
  then validating the dig-sig in the context of that specific DOM and the
  objects that were in effect at that instance in time.}

\todo{digital signatures \emph{break} the
  update abstraction [the abstraction that renderers can ignore the XRefs and
    updates and need only access the DOM]
}

\subsection{Assessments}

Currently this specification adheres to the PDF 2.0 standard but this
limits its use, as so many PDF tools allow multiple variances from the
standard.
%
However, the opportunity is for us to use our specification to encode
some of these common extensions and to explore if the combination of
these extensions are truly unambiguous.

With our specification based approach, it would not be onerous
to refactor our specification to support different PDF variations:
\begin{itemize}
\item strict PDF-2.0.
\item strict PDF-2.0, validating everything.
\item PDF-2.0 with some common extensions that are determined to be unambiguous.
\end{itemize}

The conciseness of our specification is due to a few factors
\begin{enumerate}
\item our intention to make it as clear as possible,
\item ignoring some ``engineering'' aspects (e.g., error
   messages, recovery, etc.)
\item leaving out code for some of the simple, tedious functions.
\end{enumerate}
One line of future work would be to turn the specification into a
\emph{reference implementation} by the addition of
(1) code for the current, missing ``tedious parts'',
(2) links to implementations of the primitive parsers, and
(3) output of the DOM.

Future work might also involve further validation checks such as these:
(1) processing linearization data and ensuring consistency with the
non-linearized DOM;
(2) processing both ``branches'' of a hybrid PDF and ensuring
some form of constency between pre 1.5 readers and 1.5+ readers;
(3) adding signature validation, see \cref{sec:updates-and-signatures}.


% ------------------------------------------------------------------------------
\section{Securing the Pre-DOM Components \note{1pp}}
\label{sec:securing}

There are four primary approaches we are taking to improve the security of the
Pre-DOM components of PDF parsing:
\begin{enumerate}
\item
  \emph{Develop tools for inspection and validation of pre-DOM structures.}
  We have developed a tool that ..., this tool validates more of the structure
  than ... .  Using it we can find dead objects and cavities.
  This tool is open source and can be found as part of the Daedalus
  project \cite{daedalus-url}.
 
\item
  \emph{Understand and clarify the PDF standard.}
  In the process of doing the above, we have pushed the bounds
  of PDF knowledge and have discovered multiple places in which
  the PDF specification is ambiguous or could be made more clear.
  We have also found instances in which the specification
  does not correctly specify a well-defined parseable format:
  \todo{examples being ... trailer /Size key, inferred semantics for the XRefStm Extends key, }
  ... or in which the intentions of the committee was not made clear in
  the spec.
  
\item
  \emph{Write a formal specification for pre-DOM processing.}
  Wanting to formally capture what we have learned in 2 ...
  The ISO Standard PDF 2.0 Specification \cite{isotc171sc2wg8ISO32000220202020}
  is English, not ``formal'',
  unclear in places, \todo{...}
  (see \cref{sec:specifying}).
  
\item
  \emph{Analyze Extant Data}
  \todo{...}
  
\end{enumerate}

In this paper we won't go into further details of our tool (1).
The work done with respect to (4) is discussed further
in \cite{icarus1,icarus2}.
  % I.e., the papers from Galois & other TA1 groups.
The lessons learned in (2)
have been captured in our specification (3) as much as possible.
Thus, the remainder of our paper will be going into much greater detail of
our specification.

\section{Related Work}
\label{sec:rel-work}
% defining the PDF standard:
The PDF Association maintains the primary effort to specify the PDF
format, which is done semi-formally in the PDF standard ISO 32000, and has now
been backed partially be a machine-readable definition of the Document
Object Model~\cite{wyatt2021work}.
%
The Caradoc project aimed to define a machine-readable specification
of PDF with a verified parser, using the Coq interactive theorem
prover~\cite{g.endignouxCaradocPragmaticApproach2016}.
%
Caradoc demonstrated that a small---but meaningful---core subset could
be formalized in Coq;
%
however its definition does not include a complete definition of many
of the key features used to define referential context, including
cross-reference streams and incremental updates.

\todo{do we want to mention that the vast majority of papers on PDF security focus on DOM objects so that this is as understudied area?}
%
A wealth of recent work has discovered methods for subverting the end-to-end
guarantees of document integrity that are provided by digital
signatures in
PDF~\cite{mullerPracticalDecryptionExFiltration2019,mladenovTrillionDollarRefund2019,mullerProcessingDangerousPaths2021,ndsssymposiumNDSS2021Shadow2021,rohlmannBreakingSpecificationPDF2021}.
%
Many of these attacks rely on mutating the referential structure of a
document by updating its cross reference table via incremental
updates;
%
to our knowledge, the attacks published to date only perform basic
updates to the table, and do not rely on nuanced features of semantics
of cross-reference streams.

% parser combinators:
Parser combinators are a well-known abstraction for constructing
parsers programmatically and are available as libraries for several
industrial-strength languages, including the \texttt{parsec}
library~\cite{leijen2001parsec} originally implemented for Haskell and
ported to go, F\#, C\#, and Java;
%
the \texttt{num} parser combinator library in Rust can be used to
parse messages with \emph{zero copies}~\cite{couprie2015nom}.
%
Several experimental data-description languages, such as
Parsley~\cite{mundkurResearchReportParsley2020} and
hammer~\cite{bratus2017curing}.
%
Many of these libraries provide constructs for explicitly capturing
and setting input streams;
%
thus, we expect that many of them could be used to define the aspects
of the PDF related to referential context that we have explored in
this work.
%
We believe that this and related aspects of the PDF definition will
provide compelling motivating case studies in the design of these
languages that will drive studies of how they can used to define the
features precisely and completely.

% ------------------------------------------------------------------------------
\section{Conclusion \note{1.5pp}}
\label{sec:conclusion}

\subsection{Contributions}

\todo{prosify the following bullet points}
\begin{lstlisting}[style=meta]
  - Clarified aspects of PDF with respect to incremental updates,
    minor parsing details.
  - Submitted a problem in the definition of cross reference streams
    (\cite{isotc171sc2wg8ISO32000220202020} Sec 7.5.8) to
    ISO via Peter Wyatt.
  - Other ``submissions'' to ISO via Peter ...
  - Pre-DOM specification
    - dealing with parsing (DOM-dependent) object streams
      (the lack of clarity as submitted to ISO)
  - Tool for inspecting and checking PDF at the pre-DOM level:
    Created tool for exploring the DOM Antecedent structures
    as well as validating them (more than a
    PDF reader necessarily does).
    - Based on Galois's \todo{TA2} PDF parser, this tool can
      parse and validate each incremental update separately
      display "incremental updates," "incremental xref tables,"
      parsed objects, and cavities (bytes that are not used)
      validate that object definitions do not overlap (in their source bytes)
  - Generic and principled approach for dealing with
    \note{complex,onery,etc} formats.
    - E.g., cavities, embedded offsets/lengths, etc
\end{lstlisting}

\subsection{Applications to Other Formats}

\todo{...}

\subsection{Future Work}

Support more features in our specification:
\begin{lstlisting}[style=meta]
  - linearized PDFs
  - specify hybrid xref PDFs
  - add further ``validators''
    - no cavities
    - no dead objects
    - etc.
  - add ``Phase X'' DOM validation
\end{lstlisting}

Add more features to our tool and specification:
\begin{lstlisting}[style=meta]
  - support linearized files (to improve cavity detection)
  - more consistency checks: e.g, for hybrid xref files
  - further analysis and categorization of cavities
\end{lstlisting}

\todo{And even more?}


% ------------------------------------------------------------------------------
\section*{Acknowledgements}

This research was supported by the SafeDocs program under HR0011-19-C-0073 and HR001119C0079.
\todo{check these! PDFa number is correct}

% ----------------------------------------------------------------------------
\bibliographystyle{plain}

\bibliography{zotero-pdf-biblio,old,local}
  % zotera-pdf-biblio.bib
  %   The Zotero export (using Better BibTex Addon) of the PDF collection
  % old.bib
  %  - TA1's last LangSec paper
  % local.bib
  %  - all other references here

\end{document}
