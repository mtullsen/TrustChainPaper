%%%% listings pkg: %%%%

\lstloadlanguages{Haskell}
\lstset{
  basicstyle=\scriptsize\ttfamily,
  %mathescape=true,
  %numbers=left,
  aboveskip=\medskipamount,
  belowskip=\smallskipamount,
  keywordstyle=,
   numberstyle=\tiny,
  %stepnumber=2,
  numbersep=10pt,
  tabsize=2,
  %extendedchars=true,
  breaklines=false,
  %keywordstyle=\color{black}
  }
\newcommand{\lstcd}[1]{\lstinline[basicstyle=\ttfamily\footnotesize]{#1}}

% this style is used for long/verbatim notes (not text) in the paper
\lstdefinestyle{meta}
  {basicstyle=\scriptsize\ttfamily\color{magenta}}

\lstnewenvironment{code}
    {\lstset{}%
      \csname lst@SetFirstLabel\endcsname}
    {\csname lst@SaveFirstLabel\endcsname}
    \lstset{
      mathescape=false,
      language=Haskell,
      lineskip=1pt,
      keywordstyle=,
      basicstyle=\scriptsize\ttfamily,
      flexiblecolumns=false,
      basewidth={0.5em,0.45em},
      literate={+}{{$+$}}1 {/}{{$/$}}1 {*}{{$*$}}1 {=}{{$=$}}1
               {>}{{$>$}}1 {<}{{$<$}}1 {\\}{{$\lambda$}}1
               {\\\\}{{\char`\\\char`\\}}1
               {->}{{$\rightarrow$}}2 {>=}{{$\geq$}}2 {<-}{{$\leftarrow$}}2
               {<=}{{$\leq$}}2 {=>}{{$\Rightarrow$}}2 
               {\ .}{{$\circ$}}2 {\ .\ }{{$\circ$}}2
               {>>}{{>>}}2 {>>=}{{>>=}}2
               {|}{{$\mid$}}1               
    }
    % literate haskell assumes \begin{code}

% when we don't want Haskell to execute the code in paper:
% Alts    
%   \newenvironment{codeNoExecute}{\code}{\endcode} 
%   \let\codeNoExecute\code
%   \let\endcodeNoExecute\endcode
%   the below, just duplicated from above:
\lstnewenvironment{codeNoExecute}
    {\lstset{}%
      \csname lst@SetFirstLabel\endcsname}
    {\csname lst@SaveFirstLabel\endcsname}
    \lstset{
      mathescape=false,
      language=Haskell,
      basicstyle=\scriptsize\ttfamily,
      flexiblecolumns=false,
      basewidth={0.5em,0.45em},
      literate={+}{{$+$}}1 {/}{{$/$}}1 {*}{{$*$}}1 {=}{{$=$}}1
               {>}{{$>$}}1 {<}{{$<$}}1 {\\}{{$\lambda$}}1
               {\\\\}{{\char`\\\char`\\}}1
               {->}{{$\rightarrow$}}2 {>=}{{$\geq$}}2 {<-}{{$\leftarrow$}}2
               {<=}{{$\leq$}}2 {=>}{{$\Rightarrow$}}2 
               {\ .}{{$\circ$}}2 {\ .\ }{{$\circ$}}2
               {>>}{{>>}}2 {>>=}{{>>=}}2
               {|}{{$\mid$}}1               
    }
    % literate haskell assumes \begin{code}

    
\usepackage{hyperref}
\usepackage[capitalise]{cleveref}
  % i.e., you use '\cref{x}' rather than "Figure \ref{x}"
% inlined bib file
\usepackage{filecontents}

% SYSTEM OF ANNOTATIONS:

\newcommand{\pwnote}[1]{\textcolor{blue}{[PW: #1]}}
\newcommand{\mtnote}[1]{\textcolor{BlueGreen}{[MT: #1]}}
\newcommand{\whnote}[1]{\textcolor{Cyan}{[WH: #1]}}

\newcommand{\pwtodo}[1]{\pwnote{TODO: #1}}
\newcommand{\mttodo}[1]{\mtnote{TODO: #1}}
\newcommand{\whtodo}[1]{\whnote{TODO: #1}}


\newcommand{\info}[1]{\textcolor{brown}{[[#1]]}}
\newcommand{\note}[1]{\noteYes{#1}}  
\newcommand{\noteNo}[1]{}  % way to remove all notes.
\newcommand{\noteYes}[1]{\textcolor{red}{[[#1]]}}
\newcommand{\todo}[1]{\note{TODO: #1}}
\newcommand{\todoa}[1]{\todo{[\#A]: #1}}
\newcommand{\todob}[1]{\todo{[\#B]: #1}}
\newcommand{\todoc}[1]{\todo{[\#C]: #1}}
\newcommand{\old}[1]{\textcolor{orange}{[[OLD: #1]]}}
\newcommand{\haskellnote}[1]{#1}
  % ever add formatting to? Just used to indicate notes to non-Haskellers

\newcommand{\paragraphsection}[1]{\vspace{7pt}\noindent{\textit{#1}}}
  % use this for "Initial Results" or for sections at the lowest level.
  % TODO: is the formatting sufficiently smaller than level 2 sections?
 
\sisetup{group-separator={,}, group-minimum-digits=4}

