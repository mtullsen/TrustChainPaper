%%%% listings pkg: %%%%

% \lstloadlanguages{Haskell}

\lstset{
  basicstyle=\scriptsize\ttfamily,
  %mathescape=true,
  %numbers=left,
  mathescape=false,
  aboveskip=\medskipamount,
  belowskip=\smallskipamount,
  showspaces=false,
  keywordstyle=,
  numberstyle=\tiny,
  numbersep=0pt,
  %stepnumber=2,
  tabsize=2,
  %extendedchars=true,
  breaklines=false,
  %keywordstyle=\color{black}
  }
  % TODO this inherited or not from other lstnewenvironment
  
\newcommand{\lstcd}[1]{\lstinline[basicstyle=\ttfamily\footnotesize]{#1}}

% this style is used for long/verbatim notes (not text) in the paper
\lstdefinestyle{meta}
  {basicstyle=\scriptsize\ttfamily\color{magenta}}

\lstnewenvironment{myc}
    {\lstset{}%
      \csname lst@SetFirstLabel\endcsname}
    {\csname lst@SaveFirstLabel\endcsname}
    \lstset{
      basicstyle=\scriptsize\ttfamily,
      mathescape=false,
      language=C,
      lineskip=1pt,
      showspaces=false,
      keywordstyle=,
      flexiblecolumns=false,
      basewidth={0.5em,0.45em}
    }

\lstnewenvironment{code}
    {\lstset{}%
      \csname lst@SetFirstLabel\endcsname}
    {\csname lst@SaveFirstLabel\endcsname}
    \lstset{
      mathescape=false,
      language=Haskell,
      lineskip=1pt,
      showspaces=false,
      numbersep=0pt,
      keywordstyle=,
      basicstyle=\scriptsize\ttfamily,
      flexiblecolumns=false,
      basewidth={0.5em,0.45em},
      literate={+}{{$+$}}1 {/}{{$/$}}1 {*}{{$*$}}1 {=}{{$=$}}1
               {>}{{$>$}}1 {<}{{$<$}}1 {\\}{{$\lambda$}}1
               {\\\\}{{\char`\\\char`\\}}1
               {->}{{$\rightarrow$}}2 {>=}{{$\geq$}}2 {<-}{{$\leftarrow$}}2
               {<=}{{$\leq$}}2 {=>}{{$\Rightarrow$}}2 
          %     {\ .\ }{{$\circ$}}2
               {>>}{{>>}}2 {>>=}{{>>=}}2
               {|}{{$\mid$}}1               
    }
    % literate haskell assumes \begin{code}

\newcommand{\haskellnote}[1]{#1}
  % ever add formatting to? Just used to indicate notes to non-Haskellers
\newcommand{\keypoint}[1]{{\bf{#1}}}
  % TODO: boxed frame maybe?
