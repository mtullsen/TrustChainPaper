% ------------------------------------------------------------------------------
\section{Introduction \note{1.5pp}}
\label{sec:intro}
% background: conventional grammars:
The task of parsing may be viewed as receiving a document in an
unstructured, or serialized, form, and building its structured
representation, equipped with an argument that the structured
representation is a faithful representation of the given document.
%
For formats defined in conventional data-description languages that
correspond closely to well-understood classes of the Chomsky hierarchy
(i.e., context-free grammars, with regular expressions as a critical
special case), trustworthy arguments fall out naturally from the
definition of the format itself.
%
This is in part due to the fact that in such formats, the structured
representation of a large segment of a message is defined purely in
terms of the structure of its segments.
%
E.g., the structure of a record consisting of multiple fields (e.g., a
name, age, and social security number) is defined in terms of the
correct structure of each of the constituent fields.

% real formats have DOMs, which involve names and references
However, these key properties regarding context-freedom of a
definition critically do not hold for many practical formats of
interest.
%
Many such formats define a \emph{Document Object Model}: i.e., the
result of parsing a document is a graph between objects of data, each
of which may store a large set of fields.
%
In such formats, it is infeasible to provide context-free definition
of well-formedness because data objects that must satisfy critical
relations may not occur contiguously in a document: related objects
may not form a tree-like hierarchy.
%
Such formats typically introduce a notion of \emph{naming} or
\emph{reference} by which objects may refer to each other.
%
A critical practical example of this design pattern---and the
motivating example of our work---is the document object model of the
\emph{Portable Document Format (PDF)}~\cite{isotc171sc2wg8ISO32000220202020};
%
PDF data objects include an object \emph{identifier}, by which other
objects may refer to them.

% further complications: people want to do other fancy tricks with
% context tables
In such formats, the types used for representing references and the
data structures used to represent context take on central importance.
%
The simplest definition of such a data structure may arguably amount
to including a single, contiguous definition of context.
%
While potentially straightforward, such a design would not support
requirements of many practical formats, including PDF.
%
In many contexts, a critical feature of formats is to update a
document incrementally by appending data: this is often best achieved
by allowing the table of references to itself by updated with new,
non-contiguous entries that effectively overwrite previous conflicting
entries.
%
Additionally, such tables of references may be compressed using
standard, but non-trivial, compression algorithms.

% potential solutions and why they fail:
Context-free grammars and weaker formalisms are not suitable to define
such formats in the full detail.
%
In the conventional setting in which a parser returns a semantic value
that is then potentially transformed by further computation, which may
itself be defined in an attribute grammar or parser client logic.
%
While feasible, the main limitation of such an approach is that
computation on semantic values must then itself effectively parse
unstructured data after computing partial contextual information;
%
such parsing logic is exactly what should be expressed declaratively
in a grammar and implemented by a generated parser.
%
Alternatively, in principle it could be feasible to express some
practical formats in fundamental grammar classes that are more
expressive than those traditionally used to define formats, such as
\emph{sensitive} grammars.
%
However, for such a general class, bounds on parsing performance are
too loose to ensure acceptable practical performance.

% our solution: very careful parser combinators:
This paper explores a third approach: parser combinators a data
description language that includes constructs for explicitly capturing
input and parsing a previously-captured input, combined using
semantic-level computation in Haskell.
%
This approach is explored within an industrial strength case study:
validating and parsing the reference table of the PDF DOM.
%
In general, using parser combinators is not new: such combinators are
available widely available in the distributions of modern industrial
strength languages~\todo{cite: Monadic parser combinators, Staged
  selective parser combinators, parsec}.
%
Moreover, with the recent interest in formalizing practical formats
and generating high-assurance parsers, such combinators have
specifically been applied specifically to formalize components of the
PDF standard related to referential context.
%
However, our work is unique in that, to our knowledge, it constitutes
the first attempt to use such combinators to formalize a comprehensive
set of features used in PDF to define referential context,
specifically incremental updates to reference tables and reference
table compression within cross-reference streams.
\textcolor{blue}{I think there is also an element of general file layout (such as cavities) as an incremental update implies it is actually incremental and thus byte offsets have some inferred semantics (cf. shadow attacks; new objects cannot be from earlier in the file or a future incremental update)} 

% results:
The results of our case study show that such features can apparently
by formalized, although the resulting definition is somewhat more
subtle than what may be implemented based on a natural interpretation
of the PDF standard, or many extant documents. \textcolor{blue}{Should we mention about here that past work on PDF has focused on the DOM, without considering how trustworthy or integral the DOM is?} 
%
However, we suspect that there is nothing in the format definition
that requires the language of combinators that we applied: a key goal
of our work is to provide this formalized definition as a worked case
study, to be improved upon using definitions in other experimental
data definition languages as they are developed.

% further applications:
Using our comprehensive formalization of referential context, we have
developed several tools for assisting format developers to determine
if their format may suffer from sophisticated threats to security that
are intricately connected to the case of parsing and constructing object models,
specifically detecting the presence of document \emph{cavities}
segments of documents that do not affect the document's semantic
value, and thus may store content that is completely unobservable to
parser clients.
%
Such cavities are a powerful mechanism for creating \emph{polyglot
  files} (i.e., files that belong to multiple, unexpected formats),
which themselves have been used in recent critical system security
exploits \todo{cite iOS vuln}.

\todo{short paragraph on PDF and its challenges}
\begin{lstlisting}[style=meta]
- Creation of DOM ("document object model")
  - list of object definitions
  - ... containing object references
  - designated root object
- Cross-reference (xref) table
  - table with byte offset for each object
- Cross-reference streams (added in PDF 1.5) [7 pp. in spec]
  - compressed, complicated, space-efficient, ...
  - allows for hybrid files with both traditional
    xref tables and new xref streams
- Incremental updates
  - by only appending to PDF file we can add, update, delete, restore objects
- Linearized PDF (efficient incremental access) a.k.a. "Fast web view"
  - "differential by design"!
\end{lstlisting}
\begin{lstlisting}[style=meta]
- We must accurately create the DOM (a DAG, no cycles allowed)
  - ... while abstracting over xref tables, xref streams,
    hybrid files, incremental updates, linearization
  - ... while recovering from errors
  - ... while doing xref table reconstruction
- This is the source of errors, ambiguities, and vulnerabilities!
\end{lstlisting}

\todo{short paragraph on PDF Vulnerabilities}
\label{sec:pdf-vulnerabilities}
\begin{lstlisting}[style=meta]
- Ambiguous files
- Polyglot files
- Shadow attacks: possible because of ability to sign dead objects
  and cavities 
  \pwnote{and, depending on attack mode, referencing "dead objects" via a later
  incremental update}
\end{lstlisting}

\todo{complete}
\paragraph*{Organization} The rest of this paper is organized as
follows;
%
\Cref{sec:background} provides technical background on which our work
is based;
%
\todo{complete}
%
\Cref{sec:rel-work} reviews related work, and %
\Cref{sec:conclusion} concludes.

