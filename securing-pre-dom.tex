% ------------------------------------------------------------------------------
\section{Securing the Pre-DOM Components \note{1pp}}
\label{sec:securing}

There are four primary approaches we are taking to improve the security of the
Pre-DOM components of PDF parsing:
\begin{enumerate}
\item
  \emph{Develop tools for inspection and validation of pre-DOM structures.}
  We have developed a tool that ..., this tool validates more of the structure
  than ... .  Using it we can find dead objects and cavities.
  This tool is open source and can be found as part of the Daedalus
  project \cite{daedalus-url}.
 
\item
  \emph{Understand and clarify the PDF standard.}
  In the process of doing the above, we have pushed the bounds
  of our PDF knowledge and have discovered multiple places in which
  the PDF specification is ambiguous or could be made more clear.
  We have also found instances in which the specification
  does not correctly specify a well-defined parseable format:
  \todo{examples being ...}
  ... or in which the intentions of the committee was not made clear in
  the spec.
  
\item
  \emph{Write a formal specification for pre-DOM processing.}
  Wanting to formally capture what we have learned in 2 ...
  The ISO Standard PDF 2.0 Specification \cite{isotc171sc2wg8ISO32000220202020}
  is English, not ``formal'',
  unclear in places, \todo{...}
  (see \cref{sec:specifying}).
  
\item
  \emph{Analyze Extant Data}
  \todo{...}
  
\end{enumerate}

In this paper we won't go into further details of our tool (1).
The work done with respect to (4) is discussed further
in \cite{icarus1,icarus2}.
  % I.e., the papers from Galois & other TA1 groups.
The lessons learned in (2)
have been captured in our specification (3) as much as possible.
Thus, the remainder of our paper will be going into much greater detail of
our specification.